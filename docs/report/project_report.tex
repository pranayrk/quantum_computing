\documentclass[12pt,twoside]{report}
\usepackage[utf8]{inputenc}
\usepackage{amsmath,amsfonts,graphicx,amsthm}
\usepackage{epsfig,amssymb}
\usepackage{caption}
\usepackage{fancyhdr}
\usepackage{lipsum}
\usepackage{verbatim}
\usepackage{braket}
\usepackage{titlesec}
% \usepackage[showframe]{geometry}

\titleformat{\chapter}[display]{\normalfont\huge\bfseries\centering}{\chaptertitlename\ \thechapter}{20pt}{\Huge}    
\titlespacing*{\chapter}{0pt}{0pt}{20pt}

\newtheoremstyle{thmstyle}{10pt}{10pt}{}{}{\bfseries}{:}{.5em}{}
\theoremstyle{thmstyle}

\newtheorem{defn}{Definition}
\newtheorem*{note}{Note}
\newtheorem{thm}{Theorem}
\newtheorem{corollary}{Corollary}
\newtheorem{eg}{Example}
%\newtheorem{ex}[subsection]{Exercise}
\newtheorem{lemma}{Lemma}
\newtheorem{result}{Result}
\newtheorem{problem}{Problem}
\newtheorem{sol}{Solution}
\newtheorem*{sol*}{Solution}
\newtheorem{remark}{Remark}
\newtheorem{prop}{Proposition}

\makeatletter
\newcommand*{\rom}[1]{\expandafter\@slowromancap\romannumeral #1@}
\makeatother

\widowpenalties 1 10000
\raggedbottom
\raggedright

\setlength{\parindent}{0pt}
\setlength{\parskip}{6pt}

\setlength{\oddsidemargin}{36pt}
\setlength{\evensidemargin}{36pt}

\renewcommand{\contentsname}{\centering Contents}

\begin{document}

\begin{titlepage}
    \begin{center}
        \vspace*{0.15cm}
        \Huge{\textbf{\large Mathematics of Quantum Computing \\(Tentative Title)}}\\
        \vspace{0.25cm}
        \small
        A report submitted in partial fulfillment\\
        of the requirement for the  degree of  \\
        \vspace{0.15cm}
        \textbf{MASTER OF SCIENCE \\ IN \\ MATHEMATICS}\\
        \vspace{0.25cm}
        by\\ \vspace{0.25cm}
        \large
        \textbf{Pranay Raja Krishnan}\\
        \vspace{0.15cm}
        \large{22MMT002}\\
        \vspace{0.15cm}
        Under the guidance of\\
        \vspace{0.15cm}
        \textbf{Dr. Trivedi Harsh Chandrakant}\\

        %\vfill
        \vspace{1cm}
        \includegraphics[width=7cm]{LNMIIT.jpg}\\
        \vspace{1cm}
        \large
        \textbf{Department of Mathematics\\
        The LNM Institute of Information Technology,\\ Rupa ki Nangal, Post-Sumel, Via-Jamdoli, Jaipur, Rajasthan 302031 (INDIA).}\\%May 2019\\
        %\title{}
    \end{center}
\end{titlepage}
\pagenumbering{roman}

\newpage

\chapter*{Abstract}
\addcontentsline{toc}{chapter}{Abstract}
\lipsum[1]

\renewcommand{\sectionmark}[1]{\markright{#1}}

\chapter*{Certificate}
\addcontentsline{toc}{chapter}{Certificate}
%\markboth{Acknowledgements}{Certificate}

This is to certify that the dissertation entitled \textbf{Mathematics of Quantum Computing (Tentative Title)} submitted by \textbf{Pranay Raja Krishnan} (22MMT002) towards the partial fulfillment of the requirement for the degree of Master of Science (M.Sc) is a bonafide record of work carried out by him at the Department of Mathematics, The LNM Institute of Information Technology, Jaipur, (Rajasthan) India, during the academic session 2023-2024 under my supervision and guidance. \\
\vspace*{3cm}
\begin{flushright}
	\hfill
	{\parbox{7cm}{\textbf{Dr. Trivedi Harsh Chandrakant\\
	Assistant Professor\\
	Department of Mathematics\\
	The LNM Institute of Information Technology, Jaipur}}}
\end{flushright}

\chapter*{Acknowledgements}
\addcontentsline{toc}{chapter}{Acknowledgements}
\markboth{Acknowledgements}{Acknowledgement}

\lipsum[2]


\vspace*{3cm}
Date: \today
\hfill{}
Pranay Raja Krishnan

\chapter*{List of Notations}
\addcontentsline{toc}{chapter}{List of Notations}
\markboth{List of Notations}{List of Notations}
Unless explicitly defined the following notations are used.\\
\textbf{TODO:} Add Required Notation \\

\begin{tabular}{p{2cm}p{5cm}}
\textbf{Symbol} & \textbf{Meaning}\\
\hline
& \\
$\subseteq$ & subset or equal to \\%[3pt]
$\not\subset$ & not subset\\%[3pt]
$\supseteq$ & superset or equal to \\
$\emptyset$ & empty set\\%[3pt]
$\in$ & belongs to\\%[3pt]
$\not \in $ & does not belong to \\
$\displaystyle\prod_{i \in I}$ & product over index set $I$ \\
$\mathbb{C}$ & the set of real numbers\\%[3pt]
$\mathbb{R}$ & the set of real numbers\\%[3pt]
$\mathbb{N}$ & the set of natural numbers\\%[3pt]
%PD Matrix & Positive Definite Matrix \\
%PSD Matrix & Positive Semi Definite Matrix
\end{tabular}

\tableofcontents
\newpage
\pagenumbering{arabic}

\chapter{Qubits}

\line(1,0){360} \\

The computers we use today are based on \textbf{bits} (binary digits) each of which can represent a $0$ or $1$ state. The rules governing these bits are laid out in classical information theory and these computers can be considered equivalent to a ideal abstract computational framework - the Turing Machine.

By exploiting certain phenomena observed in the working of quantum particles, we can derive a model of a computer which can achieve results that can not be replicated efficiently on a Turing Machine. In these quantum computers, the \textbf{qubit} (quantum bit) forms the foundational unit of computing.

Many different quantum particle effects have been used in labs - photon polarization, electron spin, the state of an atom in a cavity, and even defect centers in a diamond have been leveraged to created real life implementations of qubits. We will define a qubit as a mathematical object with a certain ruleset and expect that every real-world implementation follows the working of the abstract model.

\begin{samepage}
\begin{defn}
    A complex \textbf{Hilbert space} $\mathcal{H}$ is a vector space over $\mathbb{C}$ with a positive definite inner product $\braket{} : \mathcal{H} \times \mathcal{H} \to \mathbb{C}$ defined as $(\psi,\phi) \to \braket{\psi,\phi}$ such that for all $\phi, \phi_1, \phi_2, \psi \in \mathcal{H}$ and $a, b \in \mathbb{C}$ the inner product is:
    \begin{enumerate}
        \item conjugate symmetric: $\braket{\psi,\phi} = \overline{\braket{\phi,\psi}}$
        \item positive definite: $\braket{\psi, \psi} \geq 0$ and $\braket{\psi, \psi} = 0 \iff \psi = 0$
        \item linear in first argument: $\braket{a \phi_1 + b \phi_2, \psi} = a \braket{\phi_1, \psi} + b \braket{\phi_2, \psi}$
        \item antilinear in second argument: $\braket{\psi, a \phi_1 + b \phi_2} = \overline{a} \braket{\psi,\phi_1} + \overline{b} \braket{\psi,\phi_2}$
    \end{enumerate}
    and this inner product induces a norm $||.|| : \mathcal{H} \to \mathbb{R}$ defined as $\psi \to \sqrt{\braket{\psi,\psi}}$ in which $\mathcal{H}$ is complete.
\end{defn}
\end{samepage}

\begin{samepage}
\begin{defn}
    A \textbf{qubit} is any quantum mechanical system whose state can be completely described by a unit vector in a $2$-dimensional complex Hilbert space $\mathcal{H}$ and which follows these principles: 
\begin{itemize}
    \item Principle of Superposition
    \item Principle of Entanglement
    \item Principle of Measurement
    \item Principle of Transformation
\end{itemize}
    The Hilbert space $\mathcal{H}$ is known as the \textbf{state space} and is equipped with the inner product $\braket{}$ which is defined as $\braket{\psi, \phi} = \overline{a} c + \overline{b} d$ for any $\psi = \begin{bmatrix} a\\b\end{bmatrix}, \phi = \begin{bmatrix} c \\ d \end{bmatrix} \in \mathcal{H}$. Any unit vector of $\mathcal{H}$ is called a \textbf{state vector}.
\end{defn}

The principles in the above definition will be elaborated on in the upcoming sections.
They are empirical observations of the behaviour of quantum mechanical systems and will be considered as axioms in our abstract qubit system.
\end{samepage}

\begin{defn}
    Given a matrix $A$, the \textbf{conjugate transpose} $A^\dagger$ is obtained by transposing $A$ and applying the complex conjugate of each entry.

    $A^\dagger = (\overline{A})^T = \overline{(A^T)}$ where $\overline{A}$ is the complex conjugate of $A$ and $A^T$ is the transpose of $A$.
\end{defn}

\begin{defn}
    The inner product on the quantum mechanical Hilbert space of a qubit is defined as $\braket{\psi, \phi} = \overline{a} c + \overline{b} d$ for any  $\psi = \begin{bmatrix} a\\b\end{bmatrix}, \phi = \begin{bmatrix} c \\ d \end{bmatrix} \in \mathcal{H}$. 
\end{defn}

\begin{defn}
    Any function $\phi: V \to \mathbb{F}$ from a vector space to its base field is called a \textbf{functional}. 
\end{defn}

\begin{defn}
A linear functional $\phi$ on a normed linear space $V$ is said to be \textbf{bounded} if there exists some real $M$ such that $|| \phi(v)|| \leq M ||v||$ for all $v \in V$. 
\end{defn}

\begin{note}
A linear functional $\phi$ being bounded is equivalent to $\phi$ being continuous.
\end{note}


\begin{defn}
    The set of all continuous linear functionals on a vector space $V$ is known as the \textbf{continuous dual space} of $V$.
\end{defn}

\begin{result}[Riesz' representation theorem]
    For a continuous linear functional $\phi$ on a Hilbert space $\mathcal{H}$, there exists a unique $u \in \mathcal{H}$ such that $\phi(u,v) = \braket{u,v}$ for all $v \in \mathcal{H}$. 

    This implies Hilbert spaces are conjugate isomorphic to their own conjugate dual spaces and there exists a bijection between a Hilbert space and its continuous dual space.
\end{result}

The above property of Hilbert spaces leads us to the \textbf{Dirac Bra/Ket Notation} which is widely used in quantum mechanics and quantum information theory, and which we will follow in the rest of this paper.

\begin{samepage}
\begin{result}
For a fixed $\psi \in \mathcal{H}$, consider a linear functional $f_\psi : \mathcal{H} \to \mathbb{C}$ such that $f_\psi(\phi) = \braket{\psi,\phi}$ for all $\phi \in \mathcal{H}$. Then $f_\psi$ is continuous and unique. 
\end{result}
\begin{proof}
    $f_\psi$ is continuous if and only if it is bounded. 

    The Cauchy-Schwarz inequality for inner product tells us that $|\braket{\psi,\phi}| \leq ||\psi||||\phi||$.

    This implies $|f_\psi| = |\braket{\psi,\phi}| \leq M ||\phi||$ where $M = ||\psi||$ is a fixed quantity, i.e. $|f_\psi|$ is bounded.

    Hence $f_\psi$ is continuous. 

    
    Since $f_\psi$ is a continuous linear functional, it is an element of the continuous dual space of $\mathcal{H}$ and is therefore unique by Riesz's representation theorem.
\end{proof}
\end{samepage}

\begin{note}
    Any vector $\psi \in \mathcal{H}$ will be written as $\ket{\psi}$. This is the notation for a vector in \textbf{Dirac's bra/ket notation}, and is read \textit{ket psi}.

    Consider the unique linear functional $f_\psi$ associated with a vector $\psi \in \mathcal{H}$ as defined in the above result. The linear functional $f_\psi$ will be written as $\bra{\psi}$ and is read \textit{bra psi}.

    The linear functional $\bra{\psi}$ applied on a vector $\ket{\phi}$ is written as $\braket{\psi|\phi}$ and $\braket{\psi|\phi} = f_\psi(\phi) = \braket{\psi,\phi}$ is the inner product of $\psi$ and $\phi$.
\end{note}

\begin{thm}
    For the inner product of a qubit's state space $\mathcal{H}$ defined as $\braket{\psi|\phi} = \overline{a} c + \overline{b} d$ for any $\ket{\psi} = \begin{bmatrix} a \\ b\end{bmatrix}, \ket{\phi} = \begin{bmatrix} c \\ d \end{bmatrix} \in \mathcal{H}$, the linear functional $\bra{\psi}$ has the matrix representation $\ket{\psi}^\dagger = \begin{bmatrix} \overline{a} & \overline{b} \end{bmatrix}$. 
\end{thm}
\begin{proof}

    Consider $\ket{\psi} = \begin{bmatrix} a \\ b\end{bmatrix}, \ket{\phi} = \begin{bmatrix} c \\ d \end{bmatrix} \in \mathcal{H}$.

        Then $\ket{\psi}^\dagger \ket{\phi} = \begin{bmatrix}\overline{a} & \overline{b} \end{bmatrix}  \begin{bmatrix} c \\ d \end{bmatrix} = \overline{a} c + \overline{b} d = \braket{\psi|\phi} = f_{\ket{\psi}}(\ket{\phi})$ for some continuous linear functional $f_{\ket{\psi}}: \mathcal{H} \to \mathbb{C}$.

            Since $f_{\ket\psi}$ is unique by Riesz's representation theorem, we can set $\bra{\psi} = f_{\ket\psi}(\phi) = \ket{\psi}^\dagger \ket{\phi}$ for all $\ket\phi \in \mathcal{H}$, i.e. $\bra{\psi}$ is the linear functional which has matrix representation $\ket{\psi}^\dagger$.
\end{proof}

\pagebreak 
\section{Superposition}
\begin{lemma}[Principle of Superposition]

    Suppose $\ket{\psi}$ and $\ket{\sigma}$ are two mutually orthogonal vectors in a Hilbert space $\mathcal{H}$, and $a, b \in \mathbb{C}$. 

    Then $a \ket{\psi} + b \ket{\sigma} \in \mathcal{H}$ is a valid state vector of the state space of a qubit when $|a|^2 + |b|^2 = 1$. 

    The state of the system is completely defined by its state vector which is a unit vector in the systems' state space.
\end{lemma}

A given state of the system is completely described by a \textit{unit vector} $\ket{\psi}$, which is called the \textbf{state vector} (or wave function) on the Hilbert Space. This leads to qubits being referred to as \textbf{two-state} quantum systems since its state is the linear combination of two orthogonal basis vectors. 

These orthogonal states act as the basis elements of the Hilbert space $\mathcal{H}$ modelling the qubit. When working with Hilbert spaces associated with quantum systems, we normally use \textit{orthonormal bases} to describe state vectors.

\begin{defn}
The \textbf{computational basis} for the two dimensional complex vector space $\mathcal{H}$ is $\{ \ket{0}, \ket{1} \}$ where $\ket{0} = \begin{bmatrix} 1 \\ 0 \end{bmatrix}$ and $\ket{1} = \begin{bmatrix} 0 \\ 1 \end{bmatrix}$

With respect to the computational basis $\{ \ket{0}, \ket{1} \}$, the state of the qubit can be described as \\ $\ket{\psi} = a \ket{0} + b \ket{1} = \begin{bmatrix} a \\ 0 \end{bmatrix} + \begin{bmatrix} 0 \\ b \end{bmatrix} = \begin{bmatrix} a \\ b \end{bmatrix} $ where $a, b \in \mathbb{C}$ and $|a|^2 + |b|^2 = 1$.

\end{defn}

Another commonly used orthonormal basis for the Hilbert space $\mathcal{H}$ modelling a qubit is the Hadamard Basis.

\begin{defn}
The \textbf{Hadamard Basis} for the two dimensional complex vector space $\mathcal{H}$ is  $\{ \ket{+}, \ket{-} \}$ where \\
    $\ket{+} = \displaystyle\frac{1}{\sqrt 2} (\ket{0} + \ket{1}) = \frac{1}{\sqrt{2}} \begin{bmatrix} 1 \\ 1 \end{bmatrix}$ and $\ket{-} = \displaystyle\frac{1}{\sqrt 2} (\ket{0} - \ket{1}) = \frac{1}{\sqrt{2}} \begin{bmatrix} 1 \\ -1 \end{bmatrix}$
\end{defn}

\begin{samepage}
\begin{lemma}
Consider a state $\ket{\psi} = a \ket{0} + b \ket{1}$ where $a,b \in \mathbb{C}$ and $|a|^2 + |b|^2 = 1$ and a state $\ket{\sigma} = a' \ket{0} + b' \ket{1}$ where $a',b' \in \mathbb{C}$ and $|a'|^2 + |b'|^2 = 1$. Let $a \ket{0} + b \ket{1} = c (a' \ket{0} + b' \ket{1})$ where $c \in \mathbb{C}$ is a complex number of modulus $1$, i.e. $|c| = 1$. Then $\ket\psi$ and $\ket\sigma$ represent the same state.
\end{lemma}

Therefore, not all choices of $a, b \in \mathbb{C}$ with $|a|^2 + |b|^2 = 1$ result in different quantum state vectors. 
\begin{defn}
The multiple $c \in \mathbb{C}$ with $|c| = 1$ by which two vectors representing the same quantum state vector differ is called the \textbf{global phase}. 
\end{defn}
Global phases are artefacts of the mathematical framework we are using and have no physical meaning.
\end{samepage}



\section{Entanglement}

\begin{defn}
    Let $\mathcal{H}_1$ and $\mathcal{H}_2$ be finite dimensional Hilbert spaces.
    The \textbf{tensor product} $\mathcal{H}_1 \otimes \mathcal{H}_2$ is a Hilbert space in which every element can be represented as $\ket{v_1} \otimes \ket{w_1} + \ket{v_2} \otimes \ket{w_2} \otimes ... \otimes \ket{v_k} \otimes \ket{w_k}$ where $k = \text{min}(n,m)$ and $\ket{v_i} \in \mathcal{H}_1, \ket{w_i} \in \mathcal{H}_2$ and $\otimes$ is the tensor product defined to satisfy the following properties:
    \begin{enumerate}
        \item $(\ket{v_1} + \ket{v_2}) \otimes \ket{w} = \ket{v_1} \otimes \ket{w} + \ket{v_2} \otimes \ket{w}$
        \item $\ket{v} \otimes (\ket{w_1} + \ket{w_2}) = \ket{v} \otimes \ket{w_1} + \ket{v} \otimes \ket{w_2}$
        \item $(a \cdot \ket{v}) \otimes \ket{w} = \ket{v} \otimes (a \cdot \ket{w}) = a \cdot (\ket{v} \otimes \ket{w})$
    \end{enumerate}
\end{defn}

\begin{result}
    A basis for the tensor product space is given by $\{ v_i \otimes w_j | 1 \leq i \leq n, 1 \leq j \leq m \}$
\end{result}
\begin{proof}

\end{proof}

This implies if $\ket{\psi} = \alpha_1 \ket{v_1} + \alpha_2 \ket{v_2} + ... + \alpha_n \ket{v_n}$ and $\ket{\phi} = \beta_1 \ket{w_1} + \beta_2 \ket{w_2} + ... + \beta_m \ket{w_m}$, their tensor product representation with respect to the above basis is $\ket{\psi} \otimes \ket{\phi} = \sum_{i=1}^n \sum_{j=1}^m \alpha_i \beta_j \ket{v_i} \otimes \ket{w_j}$


\begin{note}
    In dirac's bra/ket notation, the tensor product $\ket{v} \otimes \ket{w}$ of $\ket{v} \in \mathcal{H}_1$ and $\ket{w} \in \mathcal{H}_2$ is written as $\ket{vw}$ or $\ket{v}\ket{w}$
\end{note}

\begin{prop}
    Let $\{ \ket{v_i} \} \subset \mathcal{H}_1$ be an orthonormal basis in $\mathcal{H}_1$ and $\{ \ket{w_j} \} \subset \mathcal{H}_2$ be an orthonormal basis in $\mathcal{H}_2$. Then the set $\{ \ket{v_i w_j} \}$ forms an orthonormal basis in $\mathcal{H}_1 \otimes \mathcal{H}_2$ and for finite-dimensional $\mathcal{H}_1$ and $\mathcal{H}_2$, $\text{dim}(\mathcal{H}_1 \otimes \mathcal{H}_2) = \text{dim}(\mathcal{H}_1) \text{dim}(\mathcal{H}_2)$
\end{prop}
\begin{proof}
\end{proof}

\begin{samepage}
\begin{lemma}[Principle of Entanglement]
When we have two qubits being treated as a combined system, the state space of the combined system is the tensor product $\mathcal{H}_1 \otimes \mathcal{H}_2$ of the state spaces $\mathcal{H}_1, \mathcal{H}_2$ of the component qubit subsystems. 

    Similarly, for a system of $n$ interacting qubits, the state space is the tensor product $\mathcal{H}_1 \otimes \mathcal{H}_2 \otimes ... \otimes \mathcal{H}_n$ of the state spaces of the $n$ qubits taken independently.
\end{lemma}
\end{samepage}

The most natural basis for $\mathcal{H} = \mathcal{H}_1 \otimes \mathcal{H}_2 \otimes ... \otimes \mathcal{H}_n$ is constructed from the tensor products of the computational basis vectors of $\mathcal{H}_1, \mathcal{H}_2, ..., \mathcal{H}_n$. 
\begin{defn}
    For $\mathcal{H}_i$ with computational basis $\{ \ket{0}_i, \ket{1}_i \}$ where $i \in \{1, ..., n\}$, the \textbf{computational basis} for $\mathcal{H} = \mathcal{H}_1 \otimes \mathcal{H}_2 ... \otimes \mathcal{H}_n$ is given by $\{ \ket{0}_1\ket{0}_2...\ket{0}_n, \ket{0}_1\ket{0}_2...\ket{0}_{n-1}\ket{1}_n, ..., \ket{1}_1\ket{1}_2...\ket{1}_n\} = \{ \ket{00...0}, \ket{00...1}, ... , \ket{11...1} \}$.

    For simplicity, this basis is often represented by condensing the bit-string inside the ket notation as $ \{ \ket{00...0}, \ket{00...1}, ... , \ket{11...1} \} = \{ \ket{0}, \ket{1},..., \ket{2^n - 1} \}$ when the context is unambiguous.
\end{defn}

\begin{eg}
For $n=2$, the state space for $\mathcal{H} = \mathcal{H}_1 \otimes \mathcal{H}_2$ has computational basis $\{ \ket{00}, \ket{01}, \ket{10}, \ket{11} \} = \{ \ket{0}, \ket{1}, \ket{2}, \ket{3} \}$.

    Any arbitary state $\ket{\psi} \in \mathcal{H}$ can be described as $\ket{\psi} = a \ket{00} + b \ket{01} + c \ket{10} + d \ket{11} = a \ket{0} + b \ket{1} + c \ket{2} + d \ket{3}$ \\ where $a, b, c, d \in \mathbb{C}$ and $|a|^2 + |b|^2 + |c|^2 + |d|^2 = 1$.
\end{eg}



\begin{defn}
    A state $\ket{\psi} \in \mathcal{H}_1 \otimes \mathcal{H}_2 \otimes ... \otimes \mathcal{H}_n$ is said to be \textbf{entangled} if it cannot be written as a simple tensor product of states $\ket{v_1} \in \mathcal{H}_1, \ket{v_2} \in \mathcal{H}_2, ..., \ket{v_n} \in \mathcal{H}_n$. 

    If we can write $\ket{\psi} = \ket{v_1}\ket{v_2}...\ket{v_n} = \ket{v_1 v_2...v_n}$, the state is said to be \textbf{seperable}.
\end{defn}

\begin{eg}
The state $\ket{\psi_2} = \frac{1}{\sqrt{2}} (\ket{01} + \ket{11})$ of a $2$-qubit system is seperable. 

We can write $\ket{\psi_2} = \frac{1}{\sqrt{2}} (\ket{0} + \ket{1}) \otimes \ket{1}$
\end{eg}


\begin{eg}
The state $\ket{\psi} = \frac{1}{\sqrt{2}} (\ket{00} + \ket{11})$ of a $2$-qubit system is an entangled state.

    Assume that  $\ket{\psi} \frac{1}{\sqrt{2}} (\ket{00} + \ket{11}) $ can be decomposed as $\ket{\psi} = (\alpha_1 \ket{0}_1 + \beta_1 \ket{1}_1) \otimes (\alpha_2 \ket{0}_2 + \beta_2 \ket{1}_2) = \alpha_1 \alpha_2 \ket{00} + \alpha_1 \beta_2 \ket{01} + \beta_1 \alpha_2 \ket{10} + \beta_1 \beta_2 \ket{11}$.

    Equating the components, we find $\alpha_1 \alpha_2 = \displaystyle\frac{1}{\sqrt{2}}$, $\alpha_1 \beta_2 = 0$, $\beta_1 \alpha_2 = 0$ and $\beta_1 \beta_2 = \displaystyle\frac{1}{\sqrt{2}}$. These equations cannot be satisfied simulataneously as either one of $\alpha_1$ or $\beta_2$ has to be $0$.
\end{eg}

\begin{prop}
    For Hilbert spaces $\mathcal{H}_1$ and $\mathcal{H}_2$ defining qubit systems, most states in the tensor product space $\mathcal{H}_1 \otimes \mathcal{H}_2$ of the interacting qubit systems are entangled.
\end{prop}
\begin{proof}[intuition]
    Consider a state $\ket{\psi} = a \ket{0} + b \ket{1} \in \mathcal{H}_1$, $a,b \in \mathbb{C}$. Since $a$ and $b$ are complex coefficients, we would have $4$ degrees of freedom to assign a particular $\ket{\psi}$. However including the constraints that $a^2 + b^2 = 1$ and that multiplying by global phase leaves the state unchanged, we are effectively left with $2$ degrees of freedom for assigning $\ket{\psi}$. 
    
Similarly assigning $\ket{\phi} \in \mathcal{H}_2$ has $2$ degrees of freedom.

    Consider the $4$-dimensional tensor space $\mathcal{H}_1 \otimes \mathcal{H}_2$. Since the state of any vector $\ket{\omega}$ in this space can be written as $\ket{\omega} = a \ket{00} + b \ket{01} + c \ket{10} + d \ket{11}$, where $a, b, c, d \in \mathbb{C}$ we have $8$ degrees of freedom initially for assigning the vector $\ket{\omega}$. Including constraint $a^2 + b^2 + c^2 + d^2 = 1$ and that multiplying by global phase leaves the state unchanged, we have $6$ degrees of freedom in assigning the value of $\ket{\omega}$ which is $2$ degrees of freedom more than $4 = 2 \times 2$ from the individual qubits.
\end{proof}

\section{Measurement}
The principle of superposition might indicate that we can use the continuum state of single qubit to store an infinite amount of information. However, a principal of quantum mechanics states that we cannot interact with the qubit without fundamentally altering its state. To know the state stored in a qubit, we must perform a measurement which forces the state of the qubit to "collapse" into one of two \textit{preferred states}.

A naive version principle of measurement for a single qubit is stated below. We will formalize this notion and generalize it to multiple qubits.
\begin{lemma}[Principle of Measurement]
    Any measurement device that interacts with the qubit will be calibrated with a pair of orthonormal vectors called the \textbf{preferred basis}, say $\{ \ket{u}, \ket{v} \}$. If the state of the qubit with respect to the preferred basis is $\ket{\psi} = a \ket{u} + b \ket{v}$, then measurement of the qubit will yield either $\ket{u}$ with a probability of $|a|^2$ or $\ket{v}$ with a probability $|b|^2$. \\
The process of measurement leads to the quantum state vector $\ket{\psi}$ undergoing a discontinuous change which leads to the collapse of the state vector onto one of the vectors in the preferred basis.
\end{lemma}

To formalize this notion, we have two main options: projection-valued measures (PVM) and positive-operator-valued measure (POVM). We will proceed to describe PVMs here.

\begin{defn}
    An \textbf{observable} is a physically measurable quantity of a quantum system which is represented by a self-adjoint operator on the Hilbert space associated with the quantum system.
\end{defn}

TODO: Add direct product in dirac notation


\begin{lemma}
    The eigenvectors of an observable form an orthonormal basis for the Hilbert space.
\end{lemma}

\begin{lemma}
    In a qubit represented by Hilbert space $\mathcal{H}$, the possible measurement values of an observable are given by the spectrum $\sigma(A)$ of the self adjoint operator $A$ representing the observable.

    The probability $p_\psi(\lambda)$ that a quantum system in the pure state $\ket{\psi} \in \mathcal{H}$ yields the eigenvalue $\lambda$ of $A$ upon measurement is given by the projection $P_\lambda$ onto the eigenspace $\text{Eig}(A, \lambda)$ of $\lambda$ as $p_\psi(\lambda) = || P_\lambda \ket{\psi} ||^2$
\end{lemma}

\begin{lemma}[Principle of Measurement]
Any physical observable is associated with a self-adjoint operator $\mathcal{A}$ on the Hilbert space $\mathcal{H}_S$. The possible outcome of a measurement of the observable $\mathcal{A}$ is one of the eigenvalues of the operator $\mathcal{A}$. \\
Writing the eigenvalues equation, $\mathcal{A} \ket{i} = a_i \ket{i}$ where $\ket{i}$ is an orthonormal basis of eigenvectors of the operator $\mathcal{A}$, and  $\ket{\psi} = \sum_i c_i \ket{i}$,  then the probability that a measurement of the observable $\mathcal{A}$ results in the outcome $a_i$ is given by $p_i = |\braket{i|\psi}|^2 = |c_i|^2$
\end{lemma}


\begin{defn}
    A \textbf{density operator} is a positive semi-definite operator on the Hilbert space whose trace is equal to 1.
\end{defn}

\begin{lemma}
    For each measurement that can be defined, the probability distribution over the outcomes of the measurement can be computed from the density operator as defined by Born's rule:
    $P(x_i) = \text{tr}(\Pi_i \rho)$ where $\rho$ is the density operator and $\Pi_i$ is the projection operator onto the baiss vector corresponding to the measurement outcome $x_i$.
\end{lemma}

\begin{lemma}
    The expectation value of a quantum state $\rho$ is $<A> = \text{tr}(A \rho)$.
\end{lemma}

\begin{defn}
    Let $\mathcal{H}$ be a Hilbert space. We call states $\ket{\psi_1}, \ket{\psi_2}, ..., \ket{\psi_n} \in \mathcal{H}$ perfectly distinguishable if there exists a measurement system $\{ M_i \}_{i=1}^m$ with $m \geq n$ such that $|| M_j \ket{\psi_1} ||^2 = \begin{cases} 1 & \text{if } i = j \\ 0 & \text{if } i \neq j \end{cases}$
\end{defn}
Here \textit{perfectly distinguishable} means that there is some experiment or experimental setup that can distinguish between these two states, atleast in theory.
\begin{result}
    The states $\ket{\psi_1}, \ket{\psi_2}, ..., \ket{\psi_n}$ are perfectly distinguishable if and only if they are orthogonal. This result is the reason we use orthogonal basis in quantum computing.
\end{result}
`TODO: Refer Nielsen, Chuang

This property limits the amount of information that can be extracted from a qubit: a measurment yields atmost a single classical bit worth of information. In most cases, we also cannot make more than one measurement of original state of the qubit. On measurement, we have two possibilities, each corresponding to a probability of $|a|^2$ and $|b|^2$, then the total probability of the whole space will be $|a|^2 + |b|^2 = 1$, which is valid for unit vectors $\ket{\psi} = a \ket{0} + b \ket{1}$.

\begin{note}
    When $\ket{\psi} = a \ket{0} + b \ket{1} = \begin{bmatrix} a \\ b \end{bmatrix}$, then $\bra{\psi}$ is the conjugate transpose of $\ket{\psi}$ and is read as \textbf{bra psi}, $\bra{\psi} = \begin{bmatrix} \overline{a} & \overline{b}\end{bmatrix}$
\end{note}

This lets us write the inner product for $\mathcal{H}$ as:
For any $\ket{v} = \begin{bmatrix} a \\ b \end{bmatrix}, \ket{w} = \begin{bmatrix} c \\ d \end{bmatrix} \in \mathcal{H}$, the operation $\braket{v|w} = \bra{v}\ket{w} = \begin{bmatrix} \overline{a} & \overline{b} \end{bmatrix} \begin{bmatrix} c \\ d \end{bmatrix} = \overline{a} c + \overline{b} d $

We will consider the inner product as being linear in the second variable and conjugate-linear in the first variable.

\begin{remark}
If $\ket{\psi} = \begin{bmatrix} a \\ b \end{bmatrix}$, then we can show $\braket{0|\psi} = a$, $\braket{1|\psi} = b$.
Therefore we can write $\ket{\psi} = a \ket{0} + b \ket{1} = \braket{0|\psi} \ket{0} + \braket{1|\psi} \ket{1}$.
\end{remark}

\begin{remark}
The standard inner product of the $\ket{\psi} = \begin{bmatrix} a \\ b \end{bmatrix}$ with itself in the Hilbert space $\mathcal{H}$ can therefore be written as $\braket{\psi | \psi} = \bra{\psi}\ket{\psi} = \begin{bmatrix} \overline{a} & \overline{b}\end{bmatrix} \begin{bmatrix} a \\ b \end{bmatrix} = |a|^2 + |b|^2 = 1$
\end{remark}

`TODO: Proof that self-adjoint matrices represent measurement operators`
`TODO: Relation of POVM and matrices`

Let $\mathcal{H}_1$ be an $n$-dimensional vector space with basis $\alpha = \{ \ket{a_1}, \ket{a_2}, ..., \ket{a_n} \}$ and $\mathcal{H}_2$ be an $m$-dimensional vector space with basis $\beta = \{ \ket{b_1}, \ket{b_2}, ..., \ket{b_n} \}$, then the tensor product $\mathcal{H}_1 \otimes \mathcal{H}_2$ is an $nm$-dimensional space with basis elements of the form $\ket{a_i} \otimes \ket{b_j}$

\begin{note}
    In dirac's bra/ket notation, the tensor product of $\ket{v} \in \mathcal{H}_2, \ket{w} \in \mathcal{H}_2$ is $\ket{vw} = \ket{v}\ket{w} = \ket{v} \otimes \ket{w}$
\end{note}

The tensor product is defined to satisfy the following properties:
\begin{enumerate}
    \item $(\ket{v_1} + \ket{v_2} ) \ket{w} = \ket{v_1}\ket{w} + \ket{v_2}\ket{w}$
    \item $\ket{v}(\ket{w_1} + \ket{w_2}) =  \ket{v}\ket{w_1} + \ket{v}\ket{w_2}$
    \item $(a \cdot \ket{v})\ket{w} = \ket{v}(a \cdot \ket{w})  = a \cdot (\ket{v}\ket{w})$
\end{enumerate}
Every element $\ket{\sigma} \in \mathcal{H}_1 \otimes \mathcal{H}_2$ can be written as a superposition of elements of the basis $\{ \ket{a_i}\ket{bj} \}$ as $\ket{\sigma} = \alpha_{11} \ket{a_1 b_1} + \alpha_{12} \ket{a_1 b2} + ... + \alpha_{nm} \ket{a_n b_m}$.

Most elements $\ket{\sigma} \in \mathcal{H}_1 \otimes \mathcal{H}_2$ \textit{cannot} be decomposed to $\ket{\sigma} = \ket{v}\ket{w}$ where $v \in \mathcal{H}_1, w \in \mathcal{H}_2$.
`TODO: Check proof`

Here \textit{perfectly distinguishable} means that there is some experiment or experimental setup that can distinguish between these two states, atleast in theory.

\begin{defn}
            Let $\mathcal{H}$ be a Hilbert space. We call states $\ket{\psi_1}, \ket{\psi_2}, ..., \ket{\psi_n} \in \mathcal{H}$ perfectly distinguishable if there exists a measurement system $\{ M_i \}_{i=1}^m$ with $m \geq n$ such that $|| M_j \ket{\psi_1} ||^2 = \begin{cases} 1 & \text{if } i = j \\ 0 & \text{if } i \neq j \end{cases}$
\end{defn}

\begin{result}
            The states $\ket{\psi_1}, \ket{\psi_2}, ..., \ket{\psi_n}$ are perfectly distinguishable if and only if they are orthogonal. This result is the reason we use orthogonal basis in quantum computing.
\end{result}

Positive-Operator-Valued Measures (POVMs) are a further generalization of the Projection-Valued Measure (PVMs) and are described in the appendix. 

\section{Transformation}

\begin{lemma}[Principle of Transformation]
    Isolated Quantum states evolve unitarily, i.e. for an isolated system there exists a unitary matrix $U_t$ such that $\ket{\psi_t} = U_t \ket{\psi_0}$ where $\ket{\psi_0}$ is the starting state and $\ket{\psi_t}$ is the state at time $t$.
\end{lemma}

\begin{thm}
    Any change applied to a quantum state can be represented by a unitary matrix $M$.
\end{thm}
\begin{proof}
    The initial state of the quantum state is a unit and so is the result state. This means we require that the transformation applied to the unit vector $M \ket{\psi}$ is a unit vector itself.

    This will happen when $\braket{M \psi | M\psi} = 1$ for all quantum states $\ket{\psi}$

    $\implies \bra{\psi} M^\dagger M \ket{\psi} = 1$
    $\implies M^\dagger M = I$ which is the condition for $M$ being a unit vector.
\end{proof}


\begin{defn}
    An observable is a physically measurable quantity of a quantum system which is represented by a self-adjoint operator on a Hilbert space.
\end{defn}

\begin{defn}
    A matrix is said to be \textbf{unitary} if and only if one of the following conditions hold:
    \begin{enumerate}
        \item $U^\dagger U = I$
        \item $U U^\dagger = I$
        \item the columns of $U$ are orthonormal vectors
        \item the rows of $U$ are orthonormal vectors
    \end{enumerate}
\end{defn}

`Alternate from Scherer`
\begin{defn}
    An operator $U$ on $H$ is called \textbf{unitary} if \\
    $\braket{U \psi | U \phi } = \braket{\psi | \phi }$ for all $\ket{\psi}, \ket{\phi} \in \mathcal{H}$
\end{defn}

\begin{lemma}
    The operator that takes $\ket{a_1} \to \ket{b_1}$ and $\ket{a_2} \to \ket{b_2}$ is obtained by the operation:
    $\ket{b_1}\ket{b_1} + \ket{b_2}\bra{a_2}$.
\end{lemma}

\begin{lemma}
    Quantum gates have the same number of inputs and outputs.
\end{lemma}

\begin{lemma}
    Quantum Gates are reversible.
\end{lemma}

\chapter{Gates and Circuits}
\line(1,0){360} \\

\begin{prop}
Gates are described by unitary matrices.
\end{prop}
\begin{proof}
    Let $M$ be a transformation of a quantum qubit state $\ket{\psi}$. Physics says that the evolution of an isolated quantum system is linear, so the transformation $M$ can be described by a matrix. \\

    For any state $\ket{\psi}$, $M \ket{\psi}$ has to be a unit vector. \\
    If $M \ket{\psi}$ is a unit vector, the inner product with itself is $1$.\\
    $\implies \braket{ \ (M\ket{\psi}) \ |\  (M \ket{\psi}) \ } = 1$\\
    $\implies \bra{\psi} M^\dagger M \ket{\psi} = 1$ \\
    $\implies M^\dagger M = I$ where $I$ is the identity matrix\\
    $\implies$ $M$ is unitary\\
\end{proof}

\begin{eg}
    Verify that the Hadamard gate is unitary and find its eigenvalues, eigenvectors.
\end{eg}

Since $U U^\dagger = I$, we can conclude that every unitary matrix is invertible.
This leads to the following result:
\begin{result}
    Only reversible gates can be implemented in quantum computing and any reversible gate has a quantum analog.
\end{result}

This shows us that the classical $\text{NOT}$ gate has a quantum analog but $\text{NAND}$ does not.

Any unitary $2 \times 2$ matrix is a valid gate but only a few are used in practice.


`TODO Notation $\ket{}\bra{}$`
\section{Gates on a single Qubit}

\subsection{Pauli Gates}
\begin{defn}[Pauli Gates]
    $I, X, Y, Z$ are known as the Pauli gates and are defined as: 
    \begin{enumerate}
        \item $I = \begin{bmatrix} 1 & 0 \\ 0 & 1 \end{bmatrix} = \ket{0}\bra{0} + \ket{1}\bra{1}$
        \item $X = \begin{bmatrix} 0 & 1 \\ 1 & 0 \end{bmatrix} = \ket{1}\bra{0} + \ket{0}\bra{1}$
        \item $Y = \begin{bmatrix} 0 & -i \\ i & 0 \end{bmatrix} = i \ket{1}\bra{0} - i \ket{0}\bra{1}$
        \item $Z = \begin{bmatrix} 1 & 0 \\ 0 & -1 \end{bmatrix} = \ket{0}\bra{0} - \ket{1}\bra{1}$
    \end{enumerate}
\end{defn}

`TODO: Effect of the Pauli Gates on the Bloch Sphere`

\subsection{Hadamard Gate}
\begin{defn}[Hadamard Gate]
    The \textbf{Hadamard Gate} is the transformation $H: \mathcal{H} \to \mathcal{H}$ such that \\
    $H \ket{0} = \frac{1}{\sqrt{2}} (\ket{0} + \ket{1}) = \ket{+}$ 
    $H \ket{1} = \frac{1}{\sqrt{2}} (\ket{0} - \ket{1}) = \ket{-}$.  It is defined by the matrix 
    $\frac{1}{\sqrt{2}} \begin{bmatrix} 1 & 1 \\ 1 & -1 \end{bmatrix} = \ket{0}\bra{+} + \ket{1}\bra{-}$
\end{defn}
The Hadamard gate allows us to obtain a superposition state.

\begin{remark}
    The Hadamard gate is its own inverse.\\
    $H^2 = \frac{1}{\sqrt{2}} \begin{bmatrix} 1 & 1 \\ 1 & -1 \end{bmatrix} \frac{1}{\sqrt{2}} \begin{bmatrix} 1 & 1 \\ 1 & -1 \end{bmatrix} = $
    $\frac{1}{2} \begin{bmatrix}2 & 0 \\ 0 & 2 \end{bmatrix} = I$
\end{remark}
\begin{remark}
    $H =   \frac{1}{\sqrt{2}} \begin{bmatrix} 1 & 1 \\ 1 & -1 \end{bmatrix} = \frac{1}{\sqrt{2}} \left( \begin{bmatrix} 0 & 1 \\ 1 & 0 \end{bmatrix}  + \begin{bmatrix} 1 & 0 \\ 0 & -1 \end{bmatrix} \right) = \frac{1}{\sqrt{2}} (X + Z)$
\end{remark}

\subsection{Phase Gate}
\begin{defn}
    The \textbf{$z$-Phase Gate} $R_z$ defines a rotation about the $z$-axis by an angle $\theta$ on the Bloch sphere. `TODO: Bloch Sphere`\\
    It is given by $R_z= \begin{bmatrix} 1 & 0 \\ 0 & e^{i\phi} \end{bmatrix} = \ket{0}\bra{0} + e^{i\phi} \ket{1}\bra{1}$ \\

    The \textbf{$y$-Phase Gate} defines a rotation about the $y$ axis and is defined by \\
    $\begin{bmatrix} \cos \frac{\theta}{2} & - \sin \frac{\theta}{2} \\ \sin \frac{\theta}{2} & \cos \frac{\theta}{2} \end{bmatrix}$ \\
    The \textbf{$x$-Phase Gate} defines a rotation about the $x$ axis and is defined by \\
    $\begin{bmatrix} \cos \frac{\theta}{2} & - i \sin \frac{\theta}{2} \\ - i \sin \frac{\theta}{2} & \cos \frac{\theta}{2} \end{bmatrix}$
\end{defn}

\section{Gates on Multiple Qubits}

\subsection{CNOT Gate}
\begin{defn}
    The \textbf{CNOT gate} is a gate that acts on 2 qubits which flips the second bit if the first bit is in the $\ket{1}$ state.\\
    It is defined by the matrix $\text{CNOT} = \begin{bmatrix} 1 & 0 & 0 & 0 \\ 0 & 1 & 0 & 0 \\ 0 & 0 & 0 & 1 \\ 0 & 0 & 1 & 0 \end{bmatrix} = \ket{0}\bra{0} \otimes I + \ket{1}\bra{1} \otimes X$
\end{defn}
The CNOT gate allows us to obtain an entangled state.

\subsection{Toffoli Gate}
\begin{defn}
    The \textbf{Toffoli gate} is a gate that acts on 3 qubits that flips the third bit if the first two are in the $\ket{1}$ state.\\
\end{defn}

Depending on the input the Toffoli gate can function as an AND, NOT and NAND gate. Since the NAND gate is universal, the Toffoli is as well. The Toffoli gate is also unitary which means it is a valid quantum gate. This shows that every classical circuit can be implemented as a quantum circuit.

\subsection{Hadamard Transform}
\begin{defn}
    Given a register of $n$ qubits, the \textbf{Hadamard Transform} $H^{\otimes n}$ is the transformation that applies the Hadamard gate $H = \begin{bmatrix} 1 & 1 \\ 1 & -1 \end{bmatrix}$ on each of the $n$ qubits.
\end{defn}
\begin{eg}
    Consider the Hadamard transform applied on a $n$ qubit register, where each qubit is in the $\ket{0}$ state, i.e. the register is in the state $\ket{0^n}$.
    Then \[ H^{\otimes n} \ket{0^n} = \frac{1}{2^{n/2}} (\ket{0} + \ket{1})(\ket{0} + \ket{1})...(\ket{0} \ket{1}) = \frac{1}{2^{n/2}} \sum_{j=0}^{2^n-1} \ket{j} \] where $\ket{j}$ is the bitstring that represents $j$ in binary.
\end{eg}

\begin{result}
    For any arbitrary state $\ket{j}$ in an $n$ qubit register $\otimes_{i=1}^n \mathcal{H}_i$, \[ H^{\otimes n} \ket{j} = \frac{1}{2^{n/2}} \sum_{k=0}^{2^n-1} (-1)^{j.k} \ket{k}\] where $j.k$ is the dot product of $j$ and $k$.
\end{result}
`Refer Shor Lec 16`

\chapter{Algorithms}

\line(1,0){360} \\

To include:
\begin{itemize}
\item Quantum Algorithms can be abstracted as circuits, and these circuits can be represented as a product of matrices
\item For any classical circuit, there exists a quantum circuit that performs the computation with similar efficiency
\item 
\end{itemize}

A circuit model for quantum computing describes all computations in terms of a circuit composed of simple gates followed by a sequence of measurements.

The standard circuit model for quantum computation uses gates from { CNOT gate together with all single qubit transformations} and its measurements as all single qubit measurements in the standard basis, i.e. all computations in the standard basis consist of a sequence of gates that are either CNOT or single-qubit gates followed by a sequence of single-qubit measurements.

In this paper, we will focus on the class of 'black box' algorithms.

\section{Deutsch Algorithm and Deutsch-Jozsa Algorithm}
\subsection{Deutsch Algorithm}
\subsection{Deutsch-Josza Algorithm}

\section{Simon's Algorithm}

\section{Grover's Search Algorithm}
\begin{itemize}
\item Unlike Shor's does not have as impressive of an advantage over classical, but can be applied to a broader range of problems
\item Simplest General problem which this can be applied to is search
\item The problem of search is to find a string $x \in \{0,1 \}^n$ such that $f(x) = 1$ or conclude that no such string exists (Note in this definition $N = 2^n$ as there are $2^n$ possibilities
\item This problem is completely unstructured. There is no clever tricks we can use in the general case (e.g. binary search if it was ordered)
\item Similar to Shor's and Simon's there will be some classical post-processing after the algorithm is run, perhaps multiple times
\item First considered by Lov Grover
\item Grover's algorithm in theory be applied to a broad range of problems (unlike Shor's) but there is question on how practical these implications are
\item Solves a black box problem, and uses a black box similar to Deutsch, Deutsch-Jozsa and Simon's Algorithm
\item One of the strengths of Grover's is that there does not need to be any promises on the black box
\item Results in a $O(\sqrt{N})$ time complexity (calls to black box) where the best classical algorithms have $O(N)$ complexity (calls to black box)
\item Depends on efficiency of black box
\item $O(\sqrt{N})$ is provably optimal, no quantum algorithm can do better
\item usually presented as a probabilistic algorithm that succeeds with high probability, but variants that do suceed with certainty are known
\item Geometric Interpretation
\item Problem Setup
\item Oracle Setup
\item Analysing Grover's Algorithm is more difficult than describing it, it will help to think about reflections and rotations in the plane
\item Shor Notes, Grover Analysis
\end{itemize}

\begin{itemize}
    \item Shor's Notes Lec 24 and 25
    \item Watrous Notes Lec 12 and 13
\end{itemize}

Example: Consider we have a an equation which has a finite set of possible solutions, each of which is numbered from $1$ to $N$, and a black box which tells us whether a particular possible solution $i \in 1, ..., N$ is a solution. A classical naive algorithm will iterate over all possibilities, plugging them one after another into the black box, and determine the solution in $O(N)$ time.

The problem is captured in a black box that is described by a Boolean function, $P : \{ 0, ..., N-1 \} \to \{ 0, 1 \}$. The goal is to find a solution $x \in \{0, ..., N-1 \}$ such that $P(x) = 1$. 
The predicate $P$ is viewed as a black box, around which we will wrap a phase oracle. For a single solution case, even the best classical algorithms must inspect $N/2$ possiblities, i.e. $O(N)$.

We are given a phase oracle that tells us whether an input $x \in 1, ..., N-1$ is a solution.

Suppose we encode $x$ in $N-1$ qubits as binary, i.e. the combination of $\ket{0}$ and $\ket{1}$ that results in $\ket{x}$ in binary. Assume that $N$ is a power of $2$ (we can do this by adding dummy search elements until we reach a power of $2$).

Then the oracle $O_p$ applied to $\ket{x}$ will give $O_p \ket{x} = \begin{cases} -\ket{x} & \text{ if } x \text{ is a solution} \\ \ket{x} & \text{otherwise } \end{cases}$

Show that it is unitary

Construct the above phase oracle transformation with Toffoli Gates, $\sigma_x$ and $\sigma_z$ gates.

Implementing this oracle:
Start with a circuit that finds whether the input is a solution or not such that it sets the output qubit to $\ket{0}$ if the input is a solution and $\ket{1}$ on the output qubit if the input is not a solution. Apply $\sigma_z$ to this output qubit and then uncompute everything to get $\pm \ket{x}$

The oracle can be described by the circuit: 
input:
\[ \sum_x c_x \ket{x} \ket{0} \]
output: 
\[ \sum_x c_x \ket{x} \ket{P(x)} \]


Grovers algorithm iteratively increases the amplitudes $c_x$ of each $\ket{x}$ with $P(x) = 1$ so that a final measurement will return a value of $x$ of interest with high probability.


Grovers Algorithm starts with the superposition of every item in the search space, i.e. a superposition of $1, ..., N-1$, call this as $\ket\psi$, i.e. $\ket\psi = \frac{1}{\sqrt{N}} \sum_{i=1}^N \ket{i}$.


Algorithm:
Repeat a certain number of times:
\begin{enumerate}
    \item Apply the oracle $O_p$
    \item Apply the Hadamard transform $H^{\otimes n}$ 
    \item Apply the gate $2 \ket{0}\bra{0} - I$
    \item Apply the Hadamard transform $H^{\otimes n}$ 
\end{enumerate}

Refer to Shor, Lec 24 to see how to implement $2 \ket{0}\bra{0} - I$ and its effect. 

Grovers Initital Analysis: 

Note: $H^{\otimes n} \ket{0} = \ket{\psi}$
This implies $H^{\otimes n} (2 \ket{0}\bra{0} - I) H^{\otimes n} = 2 \ket{sigma}\bra{sigma} - I$
 the above (last 3 steps of the algorithm) reflects all the amplitudes around their average value.

The first step $O_p$ reflects each of the amplitudes around $0$ and the last three steps reflects each of the amplitudes about their average value.

Marked state starts off with amplitude $\frac{1}{\sqrt{N}}$. First reflection about $x$ axis takes it to $- \frac{1}{\sqrt{N}}$ and average is still $\frac{1}{\sqrt{N}}$.

After last 3 steps, marked state will be $\frac{3}{\sqrt{N}}$.

One more iteration of the algorithm gives, $\frac{5}{\sqrt{N}}$ and in general, the $k$-th iteration will have the marked state as $\frac{2k + 1}{\sqrt{N}}$.

After $\frac{1}{2} \sqrt{N} \sim O(\sqrt{N})$ steps, almost all the amplitudes will be in the marked state, and we will have nearly a probability 1 of finding it.


Check Shor Lec 24 for textbook analysis of the algorithm










\newpage
\addcontentsline{toc}{chapter}{Bibliography}
\begin{thebibliography}{9}
    \bibitem{latexcompanion}
        \lipsum[1][1-3]
    \bibitem{latexcompanion}
        \lipsum[2][1-3]
    \bibitem{latexcompanion}
        \lipsum[3][1-3]
    \bibitem{latexcompanion}
        \lipsum[4][1-3]
    \bibitem{latexcompanion}
        \lipsum[5][1-3]
    \bibitem{latexcompanion}
        \lipsum[6][1-3]
\end{thebibliography}
\end{document}

