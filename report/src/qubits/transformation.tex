\section{Transformation}

\begin{comment}
    * Unitary transformation
    * Unitary matrix and properties
    * Every transformation is unitary
    * No cloning principle, Rieffel Polak P73
    * A measurement cannot be a transformation
\end{comment}

\begin{defn}
    A \textbf{unitary transformation} $U: \mathcal{H}_1 \to \mathcal{H}_2$ between two Hilbert space $\mathcal{H}_1$ and $\mathcal{H}_2$ is a isomorphism that preserves the inner product.

    For a unitary transformation $U$ on $\mathcal{H}$ we have
    $\braket{U \psi | U \phi } = \braket{\psi | \phi }$ for all $\ket{\psi}, \ket{\phi} \in \mathcal{H}$

\end{defn}

\begin{prop}
    Unitary transformations map orthonormal bases to orthonormal bases.
\end{prop}
\begin{proof}
    Consider an $n$ dimensional Hilbert space $\mathcal{H}$ with orthonormal bases $\{ \ket{e_i} \}_{i=1}^n$ and a unitary transformation $U$.
    Let $\ket{f_i} = U(\ket{e_i})$ for all $1 \leq i \leq n$.

    \emph{To show $\{ \ket{f_i} \}$ is a basis:}\\
    Consider there are constants $c_1, c_2, ..., c_n$ such that $c_1 \ket{f_1} + c_2 \ket{f_2} + ... + c_n \ket{f_n} = 0_\mathcal{H}$.

    Then $ c_1 \ket{f_1} + c_2 \ket{f_2} + ... + c_n \ket{f_n} = 0_\mathcal{H} \implies c_1 U(\ket{e_1}) + c_2 U(\ket{e_2} + ... + c_n U(\ket{e_n}) = 0_\mathcal{H} \implies U(c_1 \ket{e_1} + c_2 \ket{e_2} + ... + c_n \ket{e_n}) = 0_\mathcal{H} \implies c_1 \ket{e_1} + c_2 \ket{e_2} + ... + c_n \ket{e_n} = 0_\mathcal{H}$.

    The last implication follows from the fact that $\braket{U \psi | U \phi} = 0$ for all $\ket{\psi} \in \mathcal{H} \implies \braket{\psi|\phi} = 0$ for all $\ket\psi \in \mathcal{H} \implies \ket\phi = 0$ 

    Since $\{ \ket{e_i} \}_{i=1}^n$ is a basis, this implies $c_1 = c_2 = ...=  c_n = 0 \implies \{ \ket{f_i} \}_{i=1}^n$ is linearly independent.

    Since $\{ \ket{f_i} \}_{i=1}^n$ is a linearly independent set of $n$ elements, it is a spanning set for the $n$-dimensional space $\mathcal{H}$.

    \emph{To show $\{ \ket{f_i} \}$ is orthonormal:} \\


    Then $\braket{f_\alpha|f_\beta} = \braket{U e_\alpha|U e_\beta} = \braket{e_\alpha|e_\beta} = \delta_{\alpha, \beta}$ since $U$ is unitary.

    This implies $\{ \ket{f_i} \}_{i=1}^n$ is orthonormal.
\end{proof}

\begin{defn}
    A unitary matrix $U$ is called \textbf{unitary} if its conjugate transpose $U^\dagger$ is its inverse.

    That is, a matrix is said to be unitary if $U U^\dagger = U^\dagger U = I$.
\end{defn}

\begin{thm}
A transformation is unitary if and only if its matrix representation $U$ is a unitary matrix.
\end{thm}
\begin{proof}
    Note that $\braket{U \psi | U \phi} = \bra{U \psi} \ket{U \phi} = \ket{U \psi}^\dagger \ket{U \phi} = \ket{\psi}^\dagger U^\dagger \ket{U \phi} = \bra{\psi} U^\dagger U \ket{U\phi}$.

    \emph{To show a unitary matrix $U$ is a unitary transformation: }\\
    Let $U$ be a unitary matrix with $U\dagger U = I$ and $\ket\psi, \ket\phi \in \mathcal{H}$.

    Then $\braket{U \psi | U \phi} = \bra{\psi} U^\dagger U \ket{\phi} = \bra{\psi} \ket{\phi} = \braket{\psi|\phi} \implies $the transformation is unitary.

    \emph{To show a unitary transformation is represented by a unitary matrix: }\\
    Let $U$ be a unitary transformation with $\braket{U \psi | U \phi} = \braket{\psi|\phi}$ for all $\ket\psi, \ket\phi \in \mathcal{H}$.

    Then  $\bra{\psi} U^\dagger U \ket{\phi} = \bra{\psi} \ket{\phi}$.
    Multiplying with $\bra{\psi}^{-1}$ on the left of both sides and $\ket{\phi}^{-1}$ on the right of both sides we have the required equality $U^\dagger U = I$.
\end{proof}

Nature does not allow the state of qubits to evolve arbitrarily. Isolated quantum states which are not measured evolve unitarily.

\begin{samepage}
\begin{mdframed}
\begin{lemma}[Principle of Transformation]
    \label{transformation}
    In a quantum state space $\mathcal{H}$, every change of a quantum state over time that has not been caused by measurement is described by a unitary transformation.
    
    If $\ket{\psi}_1$ is the quantum space at time $t_1$ and $\ket{\psi}_2$ is the quantum state space at time $t_2 > t_1$, then $\ket{\psi}_2$ is described by $\ket{\psi}_2 = U \ket{\psi}_1$ where $U$ is a unitary transformation on the state space $\mathcal{H}$
\end{lemma}
\end{mdframed}
\end{samepage}

\begin{prop}
    Quantum Gates are reversible.
\end{prop}
\begin{proof}
    This is a direct consequence of the fact that the inverse of any unitary matrix is also unitary.

    Therefore $U^\dagger \cdot (U \cdot \ket\psi) = \ket\psi$ for any unitary transformation $U$.
\end{proof}

The following is a characterization of reversible gates from general information theory.
\begin{lemma}
    Quantum gates have the same number of inputs and outputs.
\end{lemma}

An effect of the linearity of any quantum state transformation $U$ results in the following principle, which has applications in quantum-based communication methods.

\begin{prop}[No Cloning Principle]
    Unknown quantum states cannot be copied or cloned.
\end{prop}
\begin{proof}
    Consider we have an unknown quantum state of a single qubit $\ket\psi \in \mathcal{H}$ which is to be copied into another qubit set to $\ket{0}$.  The combined system will have the state $\ket\psi \ket{0}$.

    Suppose there is a unitary transformation $U$ acting on the two qubits that copies $\ket\psi$ to the second qubit, i.e.
    $U(\ket\psi \ket{0}) = \ket{\psi} \ket\psi$ for any given $\ket\psi \in \mathcal{H}$
    
    Let $\ket{a}$ and $\ket{b}$ be two orthogonal states in $\mathcal{H}$.

    Then $U(\ket{a}\ket{0}) = \ket{a}\ket{a}$ and $U(\ket{b}\ket{0}) = \ket{b} \ket{b}$.

    Consider $\ket{c} = \frac{1}{\sqrt{2}} (\ket{a} + \ket{b})$

    Then $U(\ket{c}\ket{0}) = \ket{c} \ket{c} = (\frac{1}{\sqrt{2}} (\ket{a} + \ket{b}) ) (\frac{1}{\sqrt{2}} (\ket{a} + \ket{b}) = \frac{1}{2} (\ket{a}\ket{a} + \ket{a}\ket{b} + \ket{b}\ket{a} + \ket{b}\ket{b})$.

    Also, $U (\ket{c} \ket{0}) = U(\frac{1}{\sqrt{2}} (\ket{a} + \ket{b}) \ket{0}) = \frac{1}{\sqrt{2}} (\ket{a}\ket{a} + \ket{b}\ket{b})$ through the linearity of $U$ $\neq \frac{1}{2} (\ket{a}\ket{a} + \ket{a}\ket{b} + \ket{b}\ket{a} + \ket{b}\ket{b}) \implies $ which is a contradiction.

\end{proof}
