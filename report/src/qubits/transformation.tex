\section{Transformation}

\begin{defn}
    A \textbf{unitary transformation} $U: \mathcal{H}_1 \to \mathcal{H}_2$ between two Hilbert space $\mathcal{H}_1$ and $\mathcal{H}_2$ is a isomorphism that preserves the inner product.

    For a unitary transformation $U$ on $\mathcal{H}$ we have
    $\braket{U \psi | U \phi } = \braket{\psi | \phi }$ for all $\ket{\psi}, \ket{\phi} \in \mathcal{H}$

\end{defn}

\begin{defn}
    A unitary matrix $U$ is called \textbf{unitary} if its conjugate transpose $U^\dagger$ is its inverse.

    That is, a matrix is said to be unitary if $U U^\dagger = U^\dagger U = I$.
\end{defn}

\begin{thm}
A unitary transformation $U$ is represented by a unitary matrix.
\end{thm}
\begin{proof}
\end{proof}

\begin{defn}
    A matrix is said to be \textbf{unitary} if and only if one of the following conditions hold:
    \begin{enumerate}
        \item $U^\dagger U = I$
        \item $U U^\dagger = I$
        \item the columns of $U$ are orthonormal vectors
        \item the rows of $U$ are orthonormal vectors
    \end{enumerate}
\end{defn}

`Alternate from Scherer`
\begin{defn}
\end{defn}


Nature does not allow the state of qubits to evolve arbitrarily. Isolated quantum states which are not measured evolve unitarily.

\begin{samepage}
\begin{mdframed}
\begin{lemma}[Principle of Transformation]
    \label{transformation}
    In a quantum state space $\mathcal{H}$, every change of a quantum state over time that has not been caused by measurement is described by a unitary transformation.
    
    If $\ket{\psi}_1$ is the quantum space at time $t_1$ and $\ket{\psi}_2$ is the quantum state space at time $t_2 > t_1$, then $\ket{\psi}_2$ is described by $\ket{\psi}_2 = U \ket{\psi}_1$ where $U$ is a unitary transformation on the state space $\mathcal{H}$
\end{lemma}
\end{mdframed}
\end{samepage}

\begin{thm}
    Any change applied to a quantum state can be represented by a unitary matrix $M$.
\end{thm}
\begin{proof}
    The initial state of the quantum state is a unit and so is the result state. This means we require that the transformation applied to the unit vector $M \ket{\psi}$ is a unit vector itself.

    This will happen when $\braket{M \psi | M\psi} = 1$ for all quantum states $\ket{\psi}$

    $\implies \bra{\psi} M^\dagger M \ket{\psi} = 1$
    $\implies M^\dagger M = I$ which is the condition for $M$ being a unit vector.
\end{proof}


\begin{lemma}
    The operator that takes $\ket{a_1} \to \ket{b_1}$ and $\ket{a_2} \to \ket{b_2}$ is obtained by the operation:
    $\ket{b_1}\ket{b_1} + \ket{b_2}\bra{a_2}$.
\end{lemma}

\begin{lemma}
    Quantum gates have the same number of inputs and outputs.
\end{lemma}

\begin{lemma}
    Quantum Gates are reversible.
\end{lemma}
