\pagebreak 
\section{Superposition}
\begin{mdframed}
\begin{lemma}[Principle of Superposition]

    Suppose $\ket{\psi}$ and $\ket{\sigma}$ are two mutually orthogonal vectors in a Hilbert space $\mathcal{H}$, and $a, b \in \mathbb{C}$. 

    Then $a \ket{\psi} + b \ket{\sigma} \in \mathcal{H}$ is a valid state vector of the state space of a qubit when $|a|^2 + |b|^2 = 1$. 

    The state of the system is completely defined by its state vector which is a unit vector in the systems' state space.
\end{lemma}
\end{mdframed}

A given state of the system is completely described by a \textit{unit vector} $\ket{\psi}$, which is called the \textbf{state vector} (or wave function) on the Hilbert Space. This leads to qubits being referred to as \textbf{two-state} quantum systems since its state is the linear combination of two orthogonal basis vectors. 

These orthogonal states act as the basis elements of the Hilbert space $\mathcal{H}$ modelling the qubit. When working with Hilbert spaces associated with quantum systems, we normally use \textit{orthonormal bases} to describe state vectors.

\begin{defn}
The \textbf{computational basis} for the two dimensional complex vector space $\mathcal{H}$ is $\{ \ket{0}, \ket{1} \}$ where $\ket{0} = \begin{bmatrix} 1 \\ 0 \end{bmatrix}$ and $\ket{1} = \begin{bmatrix} 0 \\ 1 \end{bmatrix}$

With respect to the computational basis $\{ \ket{0}, \ket{1} \}$, the state of the qubit can be described as \\ $\ket{\psi} = a \ket{0} + b \ket{1} = \begin{bmatrix} a \\ 0 \end{bmatrix} + \begin{bmatrix} 0 \\ b \end{bmatrix} = \begin{bmatrix} a \\ b \end{bmatrix} $ where $a, b \in \mathbb{C}$ and $|a|^2 + |b|^2 = 1$.

\end{defn}

Another commonly used orthonormal basis for the Hilbert space $\mathcal{H}$ modelling a qubit is the Hadamard Basis.

\begin{defn}
The \textbf{Hadamard Basis} for the two dimensional complex vector space $\mathcal{H}$ is  $\{ \ket{+}, \ket{-} \}$ where \\
    $\ket{+} = \displaystyle\frac{1}{\sqrt 2} (\ket{0} + \ket{1}) = \frac{1}{\sqrt{2}} \begin{bmatrix} 1 \\ 1 \end{bmatrix}$ and $\ket{-} = \displaystyle\frac{1}{\sqrt 2} (\ket{0} - \ket{1}) = \frac{1}{\sqrt{2}} \begin{bmatrix} 1 \\ -1 \end{bmatrix}$
\end{defn}

\begin{samepage}
\begin{lemma}
Consider a state $\ket{\psi} = a \ket{0} + b \ket{1}$ where $a,b \in \mathbb{C}$ and $|a|^2 + |b|^2 = 1$ and a state $\ket{\sigma} = a' \ket{0} + b' \ket{1}$ where $a',b' \in \mathbb{C}$ and $|a'|^2 + |b'|^2 = 1$. Let $a \ket{0} + b \ket{1} = c (a' \ket{0} + b' \ket{1})$ where $c \in \mathbb{C}$ is a complex number of modulus $1$, i.e. $|c| = 1$. Then $\ket\psi$ and $\ket\sigma$ represent the same state.
\end{lemma}

Therefore, not all choices of $a, b \in \mathbb{C}$ with $|a|^2 + |b|^2 = 1$ result in different quantum state vectors. 
\begin{defn}
The multiple $c \in \mathbb{C}$ with $|c| = 1$ by which two vectors representing the same quantum state vector differ is called the \textbf{global phase}. 
\end{defn}
Global phases are artefacts of the mathematical framework we are using and have no physical meaning.
\end{samepage}


