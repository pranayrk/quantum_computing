\section{Entanglement}

\begin{defn}
    Let $\mathcal{H}_1$ and $\mathcal{H}_2$ be finite dimensional Hilbert spaces.
    The \textbf{tensor product} $\mathcal{H}_1 \otimes \mathcal{H}_2$ is a Hilbert space in which every element can be represented as $\ket{v_1} \otimes \ket{w_1} + \ket{v_2} \otimes \ket{w_2} \otimes ... \otimes \ket{v_k} \otimes \ket{w_k}$ where $k = \text{min}(n,m)$ and $\ket{v_i} \in \mathcal{H}_1, \ket{w_i} \in \mathcal{H}_2$ and $\otimes$ is the tensor product defined to satisfy the following properties:
    \begin{enumerate}
        \item $(\ket{v_1} + \ket{v_2}) \otimes \ket{w} = \ket{v_1} \otimes \ket{w} + \ket{v_2} \otimes \ket{w}$
        \item $\ket{v} \otimes (\ket{w_1} + \ket{w_2}) = \ket{v} \otimes \ket{w_1} + \ket{v} \otimes \ket{w_2}$
        \item $(a \cdot \ket{v}) \otimes \ket{w} = \ket{v} \otimes (a \cdot \ket{w}) = a \cdot (\ket{v} \otimes \ket{w})$
    \end{enumerate}
\end{defn}

\begin{result}
    A basis for the tensor product space is given by $\{ v_i \otimes w_j | 1 \leq i \leq n, 1 \leq j \leq m \}$
\end{result}
\begin{proof}

\end{proof}

This implies if $\ket{\psi} = \alpha_1 \ket{v_1} + \alpha_2 \ket{v_2} + ... + \alpha_n \ket{v_n}$ and $\ket{\phi} = \beta_1 \ket{w_1} + \beta_2 \ket{w_2} + ... + \beta_m \ket{w_m}$, their tensor product representation with respect to the above basis is $\ket{\psi} \otimes \ket{\phi} = \sum_{i=1}^n \sum_{j=1}^m \alpha_i \beta_j \ket{v_i} \otimes \ket{w_j}$


\begin{note}
    In dirac's bra/ket notation, the tensor product $\ket{v} \otimes \ket{w}$ of $\ket{v} \in \mathcal{H}_1$ and $\ket{w} \in \mathcal{H}_2$ is written as $\ket{vw}$ or $\ket{v}\ket{w}$
\end{note}

\begin{prop}
    Let $\{ \ket{v_i} \} \subset \mathcal{H}_1$ be an orthonormal basis in $\mathcal{H}_1$ and $\{ \ket{w_j} \} \subset \mathcal{H}_2$ be an orthonormal basis in $\mathcal{H}_2$. Then the set $\{ \ket{v_i w_j} \}$ forms an orthonormal basis in $\mathcal{H}_1 \otimes \mathcal{H}_2$ and for finite-dimensional $\mathcal{H}_1$ and $\mathcal{H}_2$, $\text{dim}(\mathcal{H}_1 \otimes \mathcal{H}_2) = \text{dim}(\mathcal{H}_1) \text{dim}(\mathcal{H}_2)$
\end{prop}
\begin{proof}
\end{proof}

\begin{samepage}
\begin{lemma}[Principle of Entanglement]
When we have two qubits being treated as a combined system, the state space of the combined system is the tensor product $\mathcal{H}_1 \otimes \mathcal{H}_2$ of the state spaces $\mathcal{H}_1, \mathcal{H}_2$ of the component qubit subsystems. 

    Similarly, for a system of $n$ interacting qubits, the state space is the tensor product $\mathcal{H}_1 \otimes \mathcal{H}_2 \otimes ... \otimes \mathcal{H}_n$ of the state spaces of the $n$ qubits taken independently.
\end{lemma}
\end{samepage}

The most natural basis for $\mathcal{H} = \mathcal{H}_1 \otimes \mathcal{H}_2 \otimes ... \otimes \mathcal{H}_n$ is constructed from the tensor products of the computational basis vectors of $\mathcal{H}_1, \mathcal{H}_2, ..., \mathcal{H}_n$. 
\begin{defn}
    For $\mathcal{H}_i$ with computational basis $\{ \ket{0}_i, \ket{1}_i \}$ where $i \in \{1, ..., n\}$, the \textbf{computational basis} for $\mathcal{H} = \mathcal{H}_1 \otimes \mathcal{H}_2 ... \otimes \mathcal{H}_n$ is given by $\{ \ket{0}_1\ket{0}_2...\ket{0}_n, \ket{0}_1\ket{0}_2...\ket{0}_{n-1}\ket{1}_n, ..., \ket{1}_1\ket{1}_2...\ket{1}_n\} = \{ \ket{00...0}, \ket{00...1}, ... , \ket{11...1} \}$.

    For simplicity, this basis is often represented by condensing the bit-string inside the ket notation as $ \{ \ket{00...0}, \ket{00...1}, ... , \ket{11...1} \} = \{ \ket{0}, \ket{1},..., \ket{2^n - 1} \}$ when the context is unambiguous.
\end{defn}

\begin{eg}
For $n=2$, the state space for $\mathcal{H} = \mathcal{H}_1 \otimes \mathcal{H}_2$ has computational basis $\{ \ket{00}, \ket{01}, \ket{10}, \ket{11} \} = \{ \ket{0}, \ket{1}, \ket{2}, \ket{3} \}$.

    Any arbitary state $\ket{\psi} \in \mathcal{H}$ can be described as $\ket{\psi} = a \ket{00} + b \ket{01} + c \ket{10} + d \ket{11} = a \ket{0} + b \ket{1} + c \ket{2} + d \ket{3}$ \\ where $a, b, c, d \in \mathbb{C}$ and $|a|^2 + |b|^2 + |c|^2 + |d|^2 = 1$.
\end{eg}



\begin{defn}
    A state $\ket{\psi} \in \mathcal{H}_1 \otimes \mathcal{H}_2 \otimes ... \otimes \mathcal{H}_n$ is said to be \textbf{entangled} if it cannot be written as a simple tensor product of states $\ket{v_1} \in \mathcal{H}_1, \ket{v_2} \in \mathcal{H}_2, ..., \ket{v_n} \in \mathcal{H}_n$. 

    If we can write $\ket{\psi} = \ket{v_1}\ket{v_2}...\ket{v_n} = \ket{v_1 v_2...v_n}$, the state is said to be \textbf{seperable}.
\end{defn}

\begin{eg}
The state $\ket{\psi_2} = \frac{1}{\sqrt{2}} (\ket{01} + \ket{11})$ of a $2$-qubit system is seperable. 

We can write $\ket{\psi_2} = \frac{1}{\sqrt{2}} (\ket{0} + \ket{1}) \otimes \ket{1}$
\end{eg}


\begin{eg}
The state $\ket{\psi} = \frac{1}{\sqrt{2}} (\ket{00} + \ket{11})$ of a $2$-qubit system is an entangled state.

    Assume that  $\ket{\psi} \frac{1}{\sqrt{2}} (\ket{00} + \ket{11}) $ can be decomposed as $\ket{\psi} = (\alpha_1 \ket{0}_1 + \beta_1 \ket{1}_1) \otimes (\alpha_2 \ket{0}_2 + \beta_2 \ket{1}_2) = \alpha_1 \alpha_2 \ket{00} + \alpha_1 \beta_2 \ket{01} + \beta_1 \alpha_2 \ket{10} + \beta_1 \beta_2 \ket{11}$.

    Equating the components, we find $\alpha_1 \alpha_2 = \displaystyle\frac{1}{\sqrt{2}}$, $\alpha_1 \beta_2 = 0$, $\beta_1 \alpha_2 = 0$ and $\beta_1 \beta_2 = \displaystyle\frac{1}{\sqrt{2}}$. These equations cannot be satisfied simulataneously as either one of $\alpha_1$ or $\beta_2$ has to be $0$.
\end{eg}

\begin{prop}
    For Hilbert spaces $\mathcal{H}_1$ and $\mathcal{H}_2$ defining qubit systems, most states in the tensor product space $\mathcal{H}_1 \otimes \mathcal{H}_2$ of the interacting qubit systems are entangled.
\end{prop}
\begin{proof}[intuition]
    Consider a state $\ket{\psi} = a \ket{0} + b \ket{1} \in \mathcal{H}_1$, $a,b \in \mathbb{C}$. Since $a$ and $b$ are complex coefficients, we would have $4$ degrees of freedom to assign a particular $\ket{\psi}$. However including the constraints that $a^2 + b^2 = 1$ and that multiplying by global phase leaves the state unchanged, we are effectively left with $2$ degrees of freedom for assigning $\ket{\psi}$. 
    
Similarly assigning $\ket{\phi} \in \mathcal{H}_2$ has $2$ degrees of freedom.

    Consider the $4$-dimensional tensor space $\mathcal{H}_1 \otimes \mathcal{H}_2$. Since the state of any vector $\ket{\omega}$ in this space can be written as $\ket{\omega} = a \ket{00} + b \ket{01} + c \ket{10} + d \ket{11}$, where $a, b, c, d \in \mathbb{C}$ we have $8$ degrees of freedom initially for assigning the vector $\ket{\omega}$. Including constraint $a^2 + b^2 + c^2 + d^2 = 1$ and that multiplying by global phase leaves the state unchanged, we have $6$ degrees of freedom in assigning the value of $\ket{\omega}$ which is $2$ degrees of freedom more than $4 = 2 \times 2$ from the individual qubits.
\end{proof}
