\pagebreak
\section{Entanglement}

\begin{comment}
x notation for tensor product of vectors
x Simple notation
x properties of tensor product of vectors
* orthonormal basis for tensor product space
* Inner product defined on space of tensor products of vectors
* Definition of tensor product of spaces -- Hilbert space
* Notation for tensor product of spaces
* Tensor product multiplication rules
* Principle of Entanglement
* Proof that most states are entangled
\end{comment}

\begin{defn}
    Let $\mathcal{H}_1$ and $\mathcal{H}_2$ be Hilbert spaces with orthonormal basis $\{ \ket{e_i} \}_{i=1}^n$ and $\{ \ket{f_j} \}_{j=1}^m$ respectively.
    The \textbf{tensor product} $\mathcal{H}_1 \otimes \mathcal{H}_2$  is an $nm$-dimensional Hilbert space with basis of the form $\{ \ket{e_i} \otimes \ket{f_j} \; | \; 1 \leq i \leq n \text{ and } 1 \leq j \leq m \}$ where $\otimes$ denotes the tensor product operation which satisfies:
    \begin{enumerate}
        \item $(a \ket\psi) \otimes \ket\phi = \ket\psi \otimes (a \ket\phi) = a (\ket\psi \otimes \ket\phi)$
        \item $a (\ket\psi \otimes \ket\phi) + b (\ket\psi \otimes \ket\phi) = (a + b) (\ket\psi \otimes \ket\phi)$
        \item $(\ket\psi_1 + \ket\psi_2) \otimes \ket\phi = \ket\psi_1 \otimes \ket\phi + \ket\psi_2 \otimes \ket\phi$
        \item $\ket\psi \otimes (\ket\phi_1 + \ket\phi_2) = \ket\psi \otimes \ket\phi_1 + \ket\psi \otimes \ket\phi_2$
    \end{enumerate}
    for any $\ket{\psi},\ket{\psi_1},\ket{\psi_2} \in \mathcal{H}_1$, $\ket{\phi}, \ket{\phi_1}, \ket{\phi_2} \in \mathcal{H}_2$ and $a, b \in \mathbb{C}$

    For any two elements in $\mathcal{H}_1 \otimes \mathcal{H}_2$ of the form $\ket{\psi_1} \otimes \ket{\phi_1}$ and $\ket{\psi_2} \otimes \ket{\phi_2}$ for some $\ket{\psi_1}, \ket{\psi_2} \in \mathcal{H}_1$ and $\ket{\phi_1}, \ket{\phi_2} \in \mathcal{H}_2$, we define the inner product $\braket{} : \mathcal{H}_1 \otimes \mathcal{H}_2 \to \mathbb{C}$ as $\braket{\; \ket{\psi_1} \otimes \ket{\phi_1} \; | \; \ket{\psi_2} \otimes \ket{\phi_2} \;} = \braket{\psi_1|\psi_2}_{\mathcal{H}_1}\braket{\phi_1|\psi_2}_{\mathcal{H}_2}$ and extend it to any pair of elements of $\mathcal{H}_1 \otimes \mathcal{H}_2$ using the linearity and conjugate linearity properties of the inner product.
\end{defn}

\begin{note}
    Given orthonormal basis $\{ \ket{e_i} \}_{i=1}^n$ for $\mathcal{H}_1$ and $\{ \ket{f_j} \}_{j=1}^m$ for $\mathcal{H}_2$, consider $\ket\psi \in \mathcal{H}_1, \ket\phi \in \mathcal{H}_2$ such that $\ket\psi = c_1 \ket{e_1} + c_2 \ket{e_2} + ... + c_n \ket{e_n}$ and $\ket{\phi} = d_1 \ket{f_1} + d_2 \ket{f_2} + ... + d_m \ket{f_m}$.

    Then using the properties of the tensor product operation, $\ket\psi \otimes \ket\phi = ( c_1 \ket{e_1} + c_2 \ket{e_2} + ... + c_n \ket{e_n} ) \otimes (d_1 \ket{f_1} + d_2 \ket{f_2} + ... + d_m \ket{f_m}) = \sum_{i=1}^n \sum_{j=1}^m c_i d_j \ket{e_i} \otimes \ket{f_j}$.

    The matrix multiplication rules for tensor product is defined analogously. 

    Let $\ket\psi = \begin{bmatrix} a \\ b \end{bmatrix} \in \mathcal{H}_1$ and $\ket\phi = \begin{bmatrix} c \\ d \end{bmatrix} \in \mathcal{H}_2$.

        Then $\ket{\psi} \otimes \ket{\phi} = \begin{bmatrix} a \\ b \end{bmatrix} \otimes \begin{bmatrix} c \\ d \end{bmatrix} = \begin{bmatrix} a \begin{bmatrix} c \\ d \end{bmatrix} \\ b \begin{bmatrix} c \\ d \end{bmatrix} \end{bmatrix} = \begin{bmatrix} ac \\ ad \\ bc \\ bd \end{bmatrix}$
\end{note}

\begin{note}
    The notation for the  tensor product $\ket\psi \otimes \ket\phi$ is often simplied as $\ket\psi \ket\phi$ or even $\ket{\psi\phi}$
\end{note}

\begin{note}
    A more formal construction of the tensor product space can be found in Appendix \ref{tensor}
\end{note}



\begin{samepage}
    \begin{mdframed}
\begin{axiom}[Principle of Entanglement]
When we have two qubits being treated as a combined system, the state space of the combined system is the tensor product $\mathcal{H}_1 \otimes \mathcal{H}_2$ of the state spaces $\mathcal{H}_1, \mathcal{H}_2$ of the component qubit subsystems. 

    Similarly, for a system of $n$ interacting qubits, the state space is the tensor product $\mathcal{H}_1 \otimes \mathcal{H}_2 \otimes ... \otimes \mathcal{H}_n$ of the state spaces of the $n$ qubits taken independently.
\end{axiom}
    \end{mdframed}
\end{samepage}


\begin{prop}
    For a $2$ qubit system, the set $\{ \ket{0} \otimes \ket{0}, \ket{0} \otimes \ket{1}, \ket{1} \otimes \ket{0}, \ket{1} \otimes \ket{1} \} = \{ \ket{00}, \ket{01}, \ket{10}, \ket{11} \}$ is an orthonormal basis for the state space $\mathcal{H} = \mathcal{H}_1 \otimes \mathcal{H}_2$.
\end{prop}
\begin{proof}
    \emph{To show  $\{ \ket{00}, \ket{01}, \ket{10}, \ket{11} \}$ is a basis: } \\

    This follows from the definition of the tensor product space as $\{ \ket{0}, \ket{1} \}$ is a basis for $\mathcal{H}_1$ and $\{ \ket{0}, \ket{1} \}$ is a basis for $\mathcal{H}_2$.

    \emph{To show $\{ \ket{00}, \ket{01}, \ket{10}, \ket{11} \}$ is an orthonormal set} \\
    $\braket{ \ket{00} | \ket{00}} = \braket{0|0} \braket{0|0} = 1 \cdot 1 = 1$, 
    $\braket{ \ket{11} | \ket{11}} = \braket{1|1} \braket{1|1} = 1 \cdot 1 = 1$
    and $\braket{ \ket{00} | \ket{11}} = \braket{0|1} \braket{1|1} = 0 \cdot 1 = 0$

    This shows that $\ket{00}$ is orthonormal to $\ket{11}$. 

    Similarly, taking the inner product for all combinations of basis elements in $\{ \ket{00}, \ket{01}, \ket{10}, \ket{11} \}$, we can see that it is an orthonormal set.
\end{proof}

\begin{defn}
    The orthonormal basis $\{ \ket{00}, \ket{01}, \ket{10}, \ket{11} \}$ for $\mathcal{H} = \mathcal{H}_1 \otimes \mathcal{H}_2$ is known as the \textbf{computational basis} for $\mathcal{H}$.

    When there is no ambiguity, the elements of this basis are often represented by replacing the bit-string by the corresponding decimal value as:

    \begin{itemize}
        \item $\ket{00} = \ket{0}$
        \item $\ket{01} = \ket{1}$
        \item $\ket{10} = \ket{2}$
        \item $\ket{11} = \ket{3}$
    \end{itemize}

\end{defn}

\begin{defn}
    For a system of two interacting qubits, the set $\{ \ket{\phi^+}, \ket{\phi^-}, \ket{\psi^+}, \ket{\psi^-} \}$ forms an orthonormal basis known as the \textbf{bell basis} where 

    $\ket{\phi^+} = \frac{1}{\sqrt{2}} (\ket{00} + \ket{11}) = \begin{bmatrix} \frac{1}{\sqrt{2}} \\ 0 \\ 0 \\ \frac{1}{\sqrt{2}} \end{bmatrix}$
    $\ket{\phi^-} = \frac{1}{\sqrt{2}} (\ket{00} - \ket{11}) = \begin{bmatrix} \frac{1}{\sqrt{2}} \\ 0 \\ 0 \\ - \frac{1}{\sqrt{2}} \end{bmatrix}$
        $\ket{\psi^+} = \frac{1}{\sqrt{2}} (\ket{01} + \ket{10}) = \begin{bmatrix} 0 \\ \frac{1}{\sqrt{2}} \\ \frac{1}{\sqrt{2}} \\0 \end{bmatrix}$
        $\ket{\psi^-} = \frac{1}{\sqrt{2}} (\ket{01} - \ket{10}) = \begin{bmatrix} 0 \\ \frac{1}{\sqrt{2}} \\ - \frac{1}{\sqrt{2}} \\0 \end{bmatrix}$
\end{defn}

\begin{defn}

    For a set of $n$ interacting qubits the \textbf{computational basis} is given by $\{ \ket{\underbrace{00...0}_{n \text{ times}}}, \ket{\underbrace{00...0}_{n-1 \text{ times}}1}, ... ,\ket{\underbrace{11...1}_{n \text{ times}}} \} = \{ \ket{0}, \ket{1}, ... , \ket{2^n - 1} \}$
\end{defn}




\begin{defn}
    A state $\ket{\psi} \in \mathcal{H}_1 \otimes \mathcal{H}_2 \otimes ... \otimes \mathcal{H}_n$ is said to be \textbf{entangled} if it cannot be written as a simple tensor product of states $\ket{v_1} \in \mathcal{H}_1, \ket{v_2} \in \mathcal{H}_2, ..., \ket{v_n} \in \mathcal{H}_n$. 

    If we can write $\ket{\psi} = \ket{v_1}\ket{v_2}...\ket{v_n} = \ket{v_1 v_2...v_n}$, the state is said to be \textbf{seperable}.
\end{defn}

\begin{eg}
The state $\ket{\psi_2} = \frac{1}{\sqrt{2}} (\ket{01} + \ket{11})$ of a $2$-qubit system is seperable since we can write $\ket{\psi_2} = \frac{1}{\sqrt{2}} (\ket{0} + \ket{1}) \otimes \ket{1}$
\end{eg}


\begin{eg}
The state $\ket{\psi} = \frac{1}{\sqrt{2}} (\ket{00} + \ket{11})$ of a $2$-qubit system is an entangled state.

    Assume that  $\ket{\psi} \frac{1}{\sqrt{2}} (\ket{00} + \ket{11}) $ can be decomposed as $\ket{\psi} = (\alpha_1 \ket{0}_1 + \beta_1 \ket{1}_1) \otimes (\alpha_2 \ket{0}_2 + \beta_2 \ket{1}_2) = \alpha_1 \alpha_2 \ket{00} + \alpha_1 \beta_2 \ket{01} + \beta_1 \alpha_2 \ket{10} + \beta_1 \beta_2 \ket{11}$.

    Equating the components, we find $\alpha_1 \alpha_2 = \displaystyle\frac{1}{\sqrt{2}}$, $\alpha_1 \beta_2 = 0$, $\beta_1 \alpha_2 = 0$ and $\beta_1 \beta_2 = \displaystyle\frac{1}{\sqrt{2}}$. These equations cannot be satisfied simulataneously as either one of $\alpha_1$ or $\beta_2$ has to be $0$.
\end{eg}


For Hilbert spaces $\mathcal{H}_1$ and $\mathcal{H}_2$ defining qubit systems, most states in the tensor product space $\mathcal{H}_1 \otimes \mathcal{H}_2$ of the interacting qubit systems are entangled.

\begin{comment}
\begin{prop}
    The set of seperable states has measure 0.
\end{prop}


\begin{proof}[intuition]
    refer https://physics.stackexchange.com/questions/268831/are-there-more-entangled-states-or-non-entangled-ones 

    Consider a state $\ket{\psi} = a \ket{0} + b \ket{1} \in \mathcal{H}_1$, $a,b \in \mathbb{C}$. Since $a$ and $b$ are complex coefficients, we would have $4$ degrees of freedom to assign a particular $\ket{\psi}$. However including the constraints that $a^2 + b^2 = 1$ and that multiplying by global phase leaves the state unchanged, we are effectively left with $2$ degrees of freedom for assigning $\ket{\psi}$. 
    
Similarly assigning $\ket{\phi} \in \mathcal{H}_2$ has $2$ degrees of freedom.

    Consider the $4$-dimensional tensor space $\mathcal{H}_1 \otimes \mathcal{H}_2$. Since the state of any vector $\ket{\omega}$ in this space can be written as $\ket{\omega} = a \ket{00} + b \ket{01} + c \ket{10} + d \ket{11}$, where $a, b, c, d \in \mathbb{C}$ we have $8$ degrees of freedom initially for assigning the vector $\ket{\omega}$. Including constraint $a^2 + b^2 + c^2 + d^2 = 1$ and that multiplying by global phase leaves the state unchanged, we have $6$ degrees of freedom in assigning the value of $\ket{\omega}$ which is $2$ degrees of freedom more than $4 = 2 \times 2$ from the individual qubits.
\end{proof}
\end{comment}
