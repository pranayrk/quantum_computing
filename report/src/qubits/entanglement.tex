\pagebreak
\section{Entanglement}

\begin{comment}
x notation for tensor product of vectors
x Simple notation
x properties of tensor product of vectors
* orthonormal basis for tensor product space
* Inner product defined on space of tensor products of vectors
* Definition of tensor product of spaces -- Hilbert space
* Notation for tensor product of spaces
* Tensor product multiplication rules
* Principle of Entanglement
* Proof that most states are entangled
\end{comment}


\begin{prop}
    Consider we have finite dimensional Hilbert spaces $\mathcal{H}_1$ and $\mathcal{H}_2$. For fixed vectors $\ket{\psi} \in \mathcal{H}_1$ and $\ket{\phi} \in \mathcal{H}_2$, define a functional $f_{\psi,\phi}: \mathcal{H}_1 \times \mathcal{H}_2 \to \mathbb{C}$ as $f_{\psi,\phi}(\ket\xi, \ket\eta) = \braket{\xi|\psi}_{\mathcal{H}_1} \braket{\eta | \phi}_{\mathcal{H}_2}$ for some $\ket{\xi} \in \mathcal{H}_1, \ket{\eta} \in \mathcal{H}_2$.

    Then the functional $f_{\psi,\phi}$ is conjugate linear and continuous.
\end{prop}
\begin{proof}

    Consider $\ket{\xi_1}, \ket{\xi_2} \in \mathcal{H}_1$ and $\ket{\eta} \in \mathcal{H}_2$. 
    \begin{align*}
        \text{Then }f_{\psi,\phi}(\ket{\xi_1} + \ket{\xi_2}, \ket{\eta}) & = \braket{\xi_1 + \xi_2 | \psi}_{\mathcal{H}_1} \braket{\eta | \phi}_{\mathcal{H}_2} 
        \\ & = (\braket{\xi_1 | \psi}_{\mathcal{H}_1}+ \braket{\xi_2 | \psi}_{\mathcal{H}_1}) \braket{\eta|\phi}_{\mathcal{H}_2}
        \\ & = \braket{\xi_1| \psi}_{\mathcal{H}_1}\braket{\eta|\phi}_{\mathcal{H}_2} + \braket{\xi_2|\psi}_{\mathcal{H}_1}\braket{\eta|\phi}_{\mathcal{H}_2}
        \\ & = f_{\psi,\phi}(\ket{\xi_1}, \ket{\eta}) + f_{\psi,\phi}(\ket{\xi_2},\ket{\eta})
    \end{align*}

    Similarly, $f_{\psi,\phi}(\ket\xi, \ket{\eta_1} + \ket{\eta_2}) = f_{\psi,\phi}(\ket\xi, \ket{\eta_1}) + f_{\psi,\phi}(\ket\xi , \ket{\eta_2})$ 
    for some $\ket\xi \in \mathcal{H}_1$ and $\ket{\eta_1}, \ket{\eta_2}  \in \mathcal{H}_2$
    \begin{align*}
    \text{Let } a \in \mathbb{C}\text{. Then }f_{\psi,\phi}(\ket{a \xi},\ket{\eta}) & = \braket{a \xi|\psi}\braket{\eta|\phi} 
        \\ & = \bra{a \xi}\ket{\psi}\braket{\eta|\phi}
        \\ & = \ket{a \xi}^\dagger \ket{\psi} \braket{\eta|\phi}
        \\ & = \overline{a} \ket{\xi}^\dagger \ket{\psi} \braket{\eta|\phi}
        \\ & = \overline{a} \braket{\xi|\psi} \braket{\eta|\phi}
    \end{align*}

    Similarly, $f_{\psi,\phi}(\ket\xi, a \ket\eta) = \overline{a}f_{\psi,\phi}(\ket\xi, \ket\eta)$

    This implies that $f_{\psi,\phi}$ is antilinear in both arguments.

    \vspace{0.5cm}
    From Theorem \ref{innerproduct:continuous}, we have that $\braket{\xi|\psi}_{\mathcal{H}_1}$ is continuous and $\braket{\eta|\phi}_{\mathcal{H}_2}$ is continuous. Therefore, $f_{\psi,\phi}(\xi, \eta) =  \braket{\xi|\psi}_{\mathcal{H}_1} \braket{\eta | \phi}_{\mathcal{H}_2}$ is continuous since product of two complex-valued continuous functions is continuous.
\end{proof}

\begin{note}
    In dirac's bra/ket notation, the functional $f_{\psi,\phi}$ is written as $\ket{\psi} \otimes \ket{\phi}$ where $\ket{\psi} \in \mathcal{H}_1$ and $\ket{\phi} \in \mathcal{H}_2$.
    For simplicity, we also write $\ket{\psi} \otimes \ket{\phi}$ as $\ket{\psi}\ket{\phi}$ or $\ket{\psi\phi}$.
\end{note}


\begin{prop}
    Consider $\ket\psi \in \mathcal{H}_1, \ket\phi \in \mathcal{H}_2$.
    The linear functional $\ket{\psi} \otimes \ket{\phi}$ defined as $[\ket\psi \otimes \ket\phi](\ket\xi,\ket\eta) = \braket{\xi|\psi}_{\mathcal{H}_1}\braket{\eta|\phi}_{\mathcal{H}_2}$  satisfies the following properties:
    \begin{enumerate}
        \item $(a \ket\psi) \otimes \ket\phi = \ket\psi \otimes (a \ket\phi) = a (\ket\psi \otimes \ket\phi)$
        \item $a (\ket\psi \otimes \ket\phi) + b (\ket\psi \otimes \ket\phi) = (a + b) (\ket\psi \otimes \ket\phi)$
        \item $(\ket\psi_1 + \ket\psi_2) \otimes \ket\phi = \ket\psi_1 \otimes \ket\phi + \ket\psi_2 \otimes \ket\phi$
        \item $\ket\psi \otimes (\ket\phi_1 + \ket\phi_2) = \ket\psi \otimes \ket\phi_1 + \ket\psi \otimes \ket\phi_2$
    \end{enumerate}
\end{prop}
\begin{proof}
    [TODO]
\end{proof}

\begin{prop}
    Given two Hilbert spaces $\mathcal{H}_1$ and $\mathcal{H}_2$, consider the set $\mathcal{G}$ of all anti-linear and continuous functionals from $\mathcal{H}_1 \times \mathcal{H}_2$ to $\mathbb{C}$.
    Then $\mathcal{G} = \{ g: \mathcal{H}_1 \times \mathcal{H}_2 \to \mathbb{C} \; | \; g \text{ is anti-linear and continuous } \}$ is a vector space over $\mathbb{C}$ with vector addition defined as $[ g_1 + g_2] (\ket{\xi}, \ket{\eta}) = g_1(\ket{\xi}, \ket{\eta}) + g_2(\ket{\xi}, \ket{\eta})$ for any $g_1, g_2 \in \mathcal{G}$ and scalar multiplication is defined as expected.
\end{prop}
\begin{proof}

    Consider any two $g_1, g_2$ in $\mathcal{G}$.
    \begin{align*}
        [g_1 + g_2](\ket\xi, \ket\eta) & = g_1(\ket\xi, \ket\eta) + g_2(\ket\xi, \ket\eta)
        \\ & \implies [g_1 + g_2] \text{ is continuous}
    \end{align*}

    Also, consider $a \in \mathbb{C}$. 
    \begin{align*}
        \text{Then }[g_1 + g_2](a \ket\xi, \ket\eta) & = g_1(a \ket\xi, \ket\eta) + g_2(a \ket\xi, \ket\eta)
        \\ & = \overline{a} g_1(\ket\xi, \ket\eta) + \overline{a} g_2(\ket\xi, \ket\eta) 
        \\ & = \overline{a} (g_1(\ket\xi, \ket\eta) + g_2(\ket\xi,\ket\eta))
        \\ & = \overline{a}[g_1 + g_2](\ket\xi, \ket\eta)
        \\ & \implies [g_1 + g_2] \text{ is anti-linear}
    \end{align*}

    Also for any functional $g$ in $\mathcal{G}$, $a \; g(\ket\xi, \ket\eta) = a \times g(\ket\xi, \ket\eta)$ is anti-linear and continuous.

    The null function $g_0 : \mathcal{H}_1 \times \mathcal{H}_2 \to \mathbb{C}, g_0(\ket\xi, \ket\eta) = 0$ for all $\ket\xi \in \mathcal{H}_1, \ket\eta \in \mathcal{H}_2$ is the null vector in $\mathcal{G}$.

    For any $g$ in $\mathcal{G}$, its inverse in $\mathcal{G}$ is $-g$.
\end{proof}

\begin{note}
    Consider the set $\mathcal{F} = \{ \ket\psi \otimes \ket\phi \; | \; \ket\psi \in \mathcal{H}_1 \text{ and } \ket\phi \in \mathcal{H}_2 \}$ where $\ket\psi \otimes \ket\phi$ is defined as previously.
    Then any $\ket\psi \otimes \ket\phi \in \mathcal{F}$ is anti-linear and continuous which implies $\mathcal{F} \subseteq \mathcal{G}$.
\end{note}

\begin{thm}
    The set $\mathcal{F} \subseteq \mathcal{G}$ forms an orthonormal basis for $\mathcal{G}$
\end{thm}
\begin{proof}
    [TODO]
\end{proof}

This implies if $\ket{\psi} = \alpha_1 \ket{v_1} + \alpha_2 \ket{v_2} + ... + \alpha_n \ket{v_n}$ and $\ket{\phi} = \beta_1 \ket{w_1} + \beta_2 \ket{w_2} + ... + \beta_m \ket{w_m}$, their tensor product representation with respect to the above basis is $\ket{\psi} \otimes \ket{\phi} = \sum_{i=1}^n \sum_{j=1}^m \alpha_i \beta_j \ket{v_i} \otimes \ket{w_j}$


\begin{samepage}
    \begin{mdframed}
\begin{lemma}[Principle of Entanglement]
When we have two qubits being treated as a combined system, the state space of the combined system is the tensor product $\mathcal{H}_1 \otimes \mathcal{H}_2$ of the state spaces $\mathcal{H}_1, \mathcal{H}_2$ of the component qubit subsystems. 

    Similarly, for a system of $n$ interacting qubits, the state space is the tensor product $\mathcal{H}_1 \otimes \mathcal{H}_2 \otimes ... \otimes \mathcal{H}_n$ of the state spaces of the $n$ qubits taken independently.
\end{lemma}
    \end{mdframed}
\end{samepage}

\begin{eg}
For $n=2$, the state space for $\mathcal{H} = \mathcal{H}_1 \otimes \mathcal{H}_2$ has computational basis $\{ \ket{00}, \ket{01}, \ket{10}, \ket{11} \} = \{ \ket{0}, \ket{1}, \ket{2}, \ket{3} \}$.

    Any arbitary state $\ket{\psi} \in \mathcal{H}$ can be described as $\ket{\psi} = a \ket{00} + b \ket{01} + c \ket{10} + d \ket{11} = a \ket{0} + b \ket{1} + c \ket{2} + d \ket{3}$ \\ where $a, b, c, d \in \mathbb{C}$ and $|a|^2 + |b|^2 + |c|^2 + |d|^2 = 1$.
\end{eg}


\begin{defn}
    A state $\ket{\psi} \in \mathcal{H}_1 \otimes \mathcal{H}_2 \otimes ... \otimes \mathcal{H}_n$ is said to be \textbf{entangled} if it cannot be written as a simple tensor product of states $\ket{v_1} \in \mathcal{H}_1, \ket{v_2} \in \mathcal{H}_2, ..., \ket{v_n} \in \mathcal{H}_n$. 

    If we can write $\ket{\psi} = \ket{v_1}\ket{v_2}...\ket{v_n} = \ket{v_1 v_2...v_n}$, the state is said to be \textbf{seperable}.
\end{defn}

\begin{eg}
The state $\ket{\psi_2} = \frac{1}{\sqrt{2}} (\ket{01} + \ket{11})$ of a $2$-qubit system is seperable. 

We can write $\ket{\psi_2} = \frac{1}{\sqrt{2}} (\ket{0} + \ket{1}) \otimes \ket{1}$
\end{eg}


\begin{eg}
The state $\ket{\psi} = \frac{1}{\sqrt{2}} (\ket{00} + \ket{11})$ of a $2$-qubit system is an entangled state.

    Assume that  $\ket{\psi} \frac{1}{\sqrt{2}} (\ket{00} + \ket{11}) $ can be decomposed as $\ket{\psi} = (\alpha_1 \ket{0}_1 + \beta_1 \ket{1}_1) \otimes (\alpha_2 \ket{0}_2 + \beta_2 \ket{1}_2) = \alpha_1 \alpha_2 \ket{00} + \alpha_1 \beta_2 \ket{01} + \beta_1 \alpha_2 \ket{10} + \beta_1 \beta_2 \ket{11}$.

    Equating the components, we find $\alpha_1 \alpha_2 = \displaystyle\frac{1}{\sqrt{2}}$, $\alpha_1 \beta_2 = 0$, $\beta_1 \alpha_2 = 0$ and $\beta_1 \beta_2 = \displaystyle\frac{1}{\sqrt{2}}$. These equations cannot be satisfied simulataneously as either one of $\alpha_1$ or $\beta_2$ has to be $0$.
\end{eg}


For Hilbert spaces $\mathcal{H}_1$ and $\mathcal{H}_2$ defining qubit systems, most states in the tensor product space $\mathcal{H}_1 \otimes \mathcal{H}_2$ of the interacting qubit systems are entangled.
\begin{prop}
    The set of seperable states has measure 0.
\end{prop}

\begin{proof}[intuition]
    refer https://physics.stackexchange.com/questions/268831/are-there-more-entangled-states-or-non-entangled-ones 

    Consider a state $\ket{\psi} = a \ket{0} + b \ket{1} \in \mathcal{H}_1$, $a,b \in \mathbb{C}$. Since $a$ and $b$ are complex coefficients, we would have $4$ degrees of freedom to assign a particular $\ket{\psi}$. However including the constraints that $a^2 + b^2 = 1$ and that multiplying by global phase leaves the state unchanged, we are effectively left with $2$ degrees of freedom for assigning $\ket{\psi}$. 
    
Similarly assigning $\ket{\phi} \in \mathcal{H}_2$ has $2$ degrees of freedom.

    Consider the $4$-dimensional tensor space $\mathcal{H}_1 \otimes \mathcal{H}_2$. Since the state of any vector $\ket{\omega}$ in this space can be written as $\ket{\omega} = a \ket{00} + b \ket{01} + c \ket{10} + d \ket{11}$, where $a, b, c, d \in \mathbb{C}$ we have $8$ degrees of freedom initially for assigning the vector $\ket{\omega}$. Including constraint $a^2 + b^2 + c^2 + d^2 = 1$ and that multiplying by global phase leaves the state unchanged, we have $6$ degrees of freedom in assigning the value of $\ket{\omega}$ which is $2$ degrees of freedom more than $4 = 2 \times 2$ from the individual qubits.
\end{proof}
