\pagebreak
\section{Entanglement}

\begin{comment}
x notation for tensor product of vectors
x Simple notation
x properties of tensor product of vectors
* orthonormal basis for tensor product space
* Inner product defined on space of tensor products of vectors
* Definition of tensor product of spaces -- Hilbert space
* Notation for tensor product of spaces
* Tensor product multiplication rules
* Principle of Entanglement
* Proof that most states are entangled
\end{comment}


A general topological result is stated below.
\begin{result}
    \label{product:continuous}
    Let $f: A \to B$ and $g: C \to D$ be continuous functions. 
    Then the map $f \times g: A \times C \to B \times D$ given by $f \times g (a x c) = f(a) \times g(c)$ for any $a \in A, c \in C$ is continuous.
\end{result}


\begin{prop}
    Consider we have finite dimensional Hilbert spaces $\mathcal{H}_1$ and $\mathcal{H}_2$. 

    For fixed vectors $\ket{\psi} \in \mathcal{H}_1$ and $\ket{\phi} \in \mathcal{H}_2$, define a functional $f_{\psi,\phi}: \mathcal{H}_1 \times \mathcal{H}_2 \to \mathbb{C}$ as $f_{\psi,\phi}(\ket\xi, \ket\eta) = \braket{\xi|\psi}_{\mathcal{H}_1} \braket{\eta | \phi}_{\mathcal{H}_2}$ for any $\ket{\xi} \in \mathcal{H}_1, \ket{\eta} \in \mathcal{H}_2$.

    Then the functional $f_{\psi,\phi}$ is conjugate linear in both variables and continuous.
\end{prop}
\begin{proof}
    \emph{To show $f_{\psi, \phi}$ is continuous:}

    From \ref{innerproduct:continuous}, we know that the inner products $\braket{\xi|\psi}_{\mathcal{H}_1}$ is a continuous function from $\mathcal{H}_1$ to $\mathbb{C}$. 
    Similarly, $\braket{\eta|\phi}_{\mathcal{H}_2}$ is a continuous function from $\mathcal{H}_2$ to $\mathbb{C}$.


    Using Result \ref{product:continuous}, $f_{\psi, \phi}(\ket\xi, \ket\eta) = \braket{\xi | \psi}_{\mathcal{H}_1} \braket{\eta|\phi}_{\mathcal{H}_2}$ is continuous at all $\ket\xi \in \mathcal{H}_1, \ket{\eta} \in \mathcal{H}_2$
    
    \emph{To show $f_{\psi, \phi}$ is conjugate linear in both variables:}
    
    Consider $\ket{\xi_1}, \ket{\xi_2} \in \mathcal{H}_1$ and $\ket{\eta} \in \mathcal{H}_2$. 
    \begin{align*}
        \text{Then }f_{\psi,\phi}(\ket{\xi_1} + \ket{\xi_2}, \ket{\eta}) & = \braket{\xi_1 + \xi_2 | \psi}_{\mathcal{H}_1} \braket{\eta | \phi}_{\mathcal{H}_2} 
        \\ & = (\braket{\xi_1 | \psi}_{\mathcal{H}_1}+ \braket{\xi_2 | \psi}_{\mathcal{H}_1}) \braket{\eta|\phi}_{\mathcal{H}_2}
        \\ & = \braket{\xi_1| \psi}_{\mathcal{H}_1}\braket{\eta|\phi}_{\mathcal{H}_2} + \braket{\xi_2|\psi}_{\mathcal{H}_1}\braket{\eta|\phi}_{\mathcal{H}_2}
        \\ & = f_{\psi,\phi}(\ket{\xi_1}, \ket{\eta}) + f_{\psi,\phi}(\ket{\xi_2},\ket{\eta})
    \end{align*}

    Similarly, $f_{\psi,\phi}(\ket\xi, \ket{\eta_1} + \ket{\eta_2}) = f_{\psi,\phi}(\ket\xi, \ket{\eta_1}) + f_{\psi,\phi}(\ket\xi , \ket{\eta_2})$ 
    for any $\ket\xi \in \mathcal{H}_1$ and $\ket{\eta_1}, \ket{\eta_2}  \in \mathcal{H}_2$
    \begin{align*}
    \text{Let } a \in \mathbb{C}\text{. Then }f_{\psi,\phi}(\ket{a \xi},\ket{\eta}) & = \braket{a \xi|\psi}\braket{\eta|\phi} 
        \\ & = \bra{a \xi}\ket{\psi}\braket{\eta|\phi}
        \\ & = \ket{a \xi}^\dagger \ket{\psi} \braket{\eta|\phi}
        \\ & = \overline{a} \ket{\xi}^\dagger \ket{\psi} \braket{\eta|\phi}
        \\ & = \overline{a} \braket{\xi|\psi} \braket{\eta|\phi}
    \end{align*}
    Similarly, $f_{\psi,\phi}(\ket\xi, a \ket\eta) = \overline{a}f_{\psi,\phi}(\ket\xi, \ket\eta)$
\end{proof}

\begin{note}
    In dirac's bra/ket notation, the functional $f_{\psi,\phi}$ is written as $\ket{\psi} \otimes \ket{\phi}$ where $\ket{\psi} \in \mathcal{H}_1$ and $\ket{\phi} \in \mathcal{H}_2$.
    For simplicity, we also write $\ket{\psi} \otimes \ket{\phi}$ as $\ket{\psi}\ket{\phi}$ or $\ket{\psi\phi}$.
\end{note}



\begin{prop}
    Given two Hilbert spaces $\mathcal{H}_1$ and $\mathcal{H}_2$, consider the set $\mathcal{G}$ of all anti-linear and continuous functionals from $\mathcal{H}_1 \times \mathcal{H}_2$ to $\mathbb{C}$.
    Then $\mathcal{G} = \{ g: \mathcal{H}_1 \times \mathcal{H}_2 \to \mathbb{C} \; | \; g \text{ is anti-linear and continuous } \}$ is a vector space over $\mathbb{C}$ with vector addition defined as $[ g_1 + g_2] (\ket{\xi}, \ket{\eta}) = g_1(\ket{\xi}, \ket{\eta}) + g_2(\ket{\xi}, \ket{\eta})$ for any $g_1, g_2 \in \mathcal{G}$ and scalar multiplication is defined as expected.
\end{prop}
\begin{proof}
    \emph{To show $\mathcal{G}$ has a zero vector:} \\

    Consider the function $0_\mathcal{G}: \mathcal{H}_1 \times \mathcal{H}_2 \to \mathbb{C}$ defined as $0_\mathcal{G}(\ket{\xi}, \ket{\eta}) = 0$

    Then $0_\mathcal{G}$ is continuous since every constant function between topological spaces is continuous.

    Consider $\ket{\xi_1}, \ket{\xi_2} \in \mathcal{H}_1$ and $\ket{\eta} \in \mathcal{H}_2$, $a,b \in \mathbb{C}$.

    Also $0_\mathcal{G}(a \ket{\xi_1} + b \ket{\xi_2}, \ket{\eta}) = 0 = (\overline{a} \cdot 0 ) + (\overline{b} \cdot 0) = \overline{a} \; 0_\mathcal{G}(\ket{\xi_1}, \ket{\eta}) + \overline{b} \; 0_\mathcal{G} (\ket{\xi_2}, \ket{\eta}) $ 
    Similarly, $0_\mathcal{G}(\ket{\xi}, a \ket{\eta_1} + b \ket{\eta_2}) = \overline{a} 0_\mathcal{G}(\ket{\xi}, \ket{\eta_1}) + \overline{b} 0_\mathcal{G}(\ket{\xi}, \ket{\eta_2})$ for any $\ket{\xi} \in \mathcal{H}_1$, $\ket{\eta_1}, \ket{\eta_2} \in \mathcal{H}_2$ and $a,b \in \mathbb{C}$.

    This implies $0_\mathcal{G}$ is conjugate linear in both variables and is continuous $\implies 0_\mathcal{G} \in \mathcal{G}$ .

    \emph{To show $\mathcal{G}$ is closed under vector addition:}\\

    Consider any two $g_1, g_2$ in $\mathcal{G}$.
    \begin{align*}
        [g_1 + g_2](\ket\xi, \ket\eta) & = g_1(\ket\xi, \ket\eta) + g_2(\ket\xi, \ket\eta)
        \\ & \implies [g_1 + g_2] \text{ is continuous, since sum of continuous functions is continuous.}
    \end{align*}

    Also, consider $a \in \mathbb{C}$. 
    \begin{align*}
        \text{Then }[g_1 + g_2](a \ket\xi, \ket\eta) & = g_1(a \ket\xi, \ket\eta) + g_2(a \ket\xi, \ket\eta)
        \\ & = \overline{a} g_1(\ket\xi, \ket\eta) + \overline{a} g_2(\ket\xi, \ket\eta) 
        \\ & = \overline{a} (g_1(\ket\xi, \ket\eta) + g_2(\ket\xi,\ket\eta))
        \\ & = \overline{a}[g_1 + g_2](\ket\xi, \ket\eta)
        \\ & \implies [g_1 + g_2] \text{ is conjugate linear in first variable}
    \end{align*}
    Similarly, $[g_1 + g_2]$ is also continuous in the second variable. 

    This implies $[g_1 + g_2]$ is continuous and conjugate linear in both variables $\implies [g_1 + g_2] \in \mathcal{G}$.



    \emph{To show $\mathcal{G}$ is closed under scalar multiplication} \\
    Consider $a \in \mathbb{C}$.

    For any functional $g$ in $\mathcal{G}$, $[a \cdot g](\ket\xi, \ket\eta) = a \cdot [ g(\ket\xi, \ket\eta) ]$ which is conjugate linear and continuous at every $\ket\xi \in \mathcal{H}_1, \ket\eta \in \mathcal{H}_2$ $\implies [a \cdot g] \in \mathcal{G}$
\end{proof}

\begin{note}
    Consider the set $\mathcal{F} = \{ \ket\psi \otimes \ket\phi \; | \; \ket\psi \in \mathcal{H}_1 \text{ and } \ket\phi \in \mathcal{H}_2 \}$ where $\ket\psi \otimes \ket\phi$ is defined as previously.

    Then any $\ket\psi \otimes \ket\phi \in \mathcal{F}$ is anti-linear and continuous which implies $\mathcal{F} \subseteq \mathcal{G}$.
\end{note}

\begin{thm}
    Consider we have Hilbert spaces $\mathcal{H}_1$ and $\mathcal{H}_2$.

    Let $\{ \ket{e_i} \}_{i=1}^n$ be an orthonormal basis for $\mathcal{H}_1$ and $\{ \ket{f_j} \}_{j=1}^m$ be an orthonormal basis for $\mathcal{H}_2$.

    Then $\{ \ket{e_i} \otimes \ket{f_j} | 1 \leq i \leq n, 1 \leq j \leq m \}$ is an orthonormal basis for $\mathcal{G} = \{ g: \mathcal{H}_1 \times \mathcal{H}_2 \to \mathbb{C} \; | \; g \text{ is conjugate linear and continuous } \}$
\end{thm}
\begin{proof}
    Let $\mathcal{E} = \{ \ket{e_i} \otimes \ket{f_j} \} | 1 \leq i \leq n, 1 \leq j \leq m \}$

    \emph{To show $\mathcal{E}$ is a spanning set for $\mathcal{G}:$} \\

    Let $g \in \mathcal{G}$ and $\ket{\xi} \in \mathcal{H}_1, \ket{\eta} \in \mathcal{H}_2$

    Since $\{ \ket{e_i} \}_{i = 1}^n$ is an orthonormal basis for $\mathcal{H}_1$, $\ket{\xi} = c_1 \ket{e_1} + c_2 \ket{e_2} + ... + c_n \ket{e_n}$ for some $c_1, c_2, ..., c_n \in \mathbb{C}$.

    Similarly, since $\{ \ket{f_j} \}_{j = 1}^m$ is an orthonormal basis for $\mathcal{H}_2$, $\ket{\eta} = d_1 \ket{f_1} + d_2 \ket{f_2} + ... + d_n \ket{f_n}$ for some $d_1, d_2, ..., d_n \in \mathbb{C}$.

    $g(\ket{\xi}, \ket{\eta}) = g(\sum_{i=1}^n c_i \ket{e_i}, \sum_{j=1}^m d_j \ket{f_j}) = \sum_{i=1}^n \sum_{j=1}^m \overline{c_i} \overline{d_j} g(\ket{e_i}, \ket{f_j})$.

    Using the fact that $c_i = \braket{e_i|\xi}$ and $d_j = \braket{f_j|\eta}$,

     $g(\ket{\xi}, \ket{\eta}) = \sum_{i=1}^n \sum_{j=1}^m \overline{\braket{e_i|\xi}} \overline{\braket{f_j|\eta}} g(\ket{e_i}, \ket{f_j}) = \sum_{i=1}^n \sum_{j=1}^m \braket{\xi|e_i} \braket{\eta|f_j}g(\ket{e_i}, \ket{f_j}) = \sum_{i=1}^n \sum_{j=1}^m [\ket{e_i} \otimes \ket{f_j}](\ket{\xi}, \ket{\eta}) g(\ket{e_i}, \ket{f_j}) = \sum_{i=1}^n \sum_{j=1}^m g_{ij} [\ket{e_i} \otimes \ket{f_j}](\ket{e_i}, \ket{f_j})$ where $g_{ij} = g(\ket{e_i}, \ket{f_j})$.

     Thus $g = \sum_{i=1}^n \sum_{j=1}^m g_{ij} \ket{e_i} \otimes \ket{f_j}$

     $\implies \text{every } g \in \mathcal{G} \text{ is a linear combination of functions in } \mathcal{E}$

     \emph{To show $\mathcal{E}$ is a linearly independent set:} \\

     Consider $g \in \mathcal{G}$ such that $g = 0_\mathcal{G}$.

     Then $g(\ket\xi, \ket\eta) = \sum_{i=1}^n \sum_{j=1}^m g_{ij} [\ket{e_i} \otimes \ket{f_j}](\ket\xi, \ket\eta) = \sum_{i=1}^n \sum_{j=1}^m g_{ij} \braket{e_i|\xi} \braket{f_j|\eta} = 0_\mathcal{G}$.

     Consider in particular, we take $\ket\xi$ and $\ket\eta$ from the set of basis vectors of $\mathcal{H}_1$ and $\mathcal{H}_2$ respectively, i.e. $(\ket\xi = \ket{e_p}$ and $\ket\eta = \ket{f_q}$ for some $1 \leq p \leq n, 1 \leq q \leq m$.

     Then $g(\ket{e_p}, \ket{e_q}) = \sum_{i=1}^n \sum_{j=1}^m g_{ij} \braket{e_i|e_p} \braket{f_j|f_q} = g_{pq}$ since $\braket{e_i|e_p} = \begin{cases} 1 & i = p \\ 0 & i \neq p \end{cases} = \delta_{i,p}$
     and $\braket{f_j|f_q} = \begin{cases} 1 & j = q \\ 0 & j \neq q \end{cases} = \delta_{j,q}$.

         $g = 0_\mathcal{H} \implies g(\ket{e_p}, \ket{e_q}) = 0 \implies g_{pq} = 0$ for any $1 \leq p \leq n, 1 \leq j \leq m$.
         Therefore $g = 0_\mathcal{H} \implies  \sum_{i=1}^n \sum_{j=1}^m g_{ij} \ket{e_i} \otimes \ket{f_j} = 0 \implies g_{ij} = 0 \;\; \forall \;\; 1 \leq i \leq n, 1 \leq j \leq m$

         $\implies \mathcal{E}$ is a linearly independent set.

         \emph{To show $\mathcal{E}$ is orthonormal:} \\
         Consider any $\ket{e_a} \otimes \ket{f_b}, \ket{e_c} \otimes \ket{f_d} \in \mathcal{E}$ for some $1 \leq a,c \leq n, 1 \leq b,d \leq m$.

         Then $\braket{\ket{e_a} \otimes \ket{f_b} \; | \; \ket{e_c} \otimes \ket{f_d}} = \braket{e_a|e_c} \braket{f_b|f_d} $

         $ = \begin{cases} 0 & a \neq c \text{ or } b \neq d \\ 1 & a = b \text{ and } c = d \end{cases} = \delta_{a,c} \delta_{b,d}$

             $\implies \mathcal{E} \text{ is orthonormal}$ 
\end{proof}

\begin{prop}
    Let $\{ e_i \}_{i=1}^n$ be an orthonormal basis for $\mathcal{H}_1$ and $\{ f_j \}_{j=1}^m$ be an orthonormal basis for $\mathcal{H}_2$.

    For $\mathcal{G} = \{ g: \mathcal{H}_1 \times \mathcal{H}_2 \to \mathbb{C} \; | \; g \text{ is conjugate linear and continuous } \}$ and $g_1, g_2 \in \mathcal{G}$ where $g_1 = \sum_{i=1}^n \sum_{j=1}^m c_{ij} \ket{e_i} \otimes \ket{f_j}$ and $g_2 = \sum_{i=1}^n \sum_{j=1}^m d_{ij} \ket{e_i} \otimes \ket{f_j}$, the function $\braket{g_1 | g_2} = \sum_{i=1}^n \sum_{j=1}^m \overline{c_{ij}} d_{ij}$ defines an inner product on $\mathcal{G}$.

\end{prop}
\begin{proof}
    $g_1 = \sum_{i=1}^n \sum_{j=1}^m c_{ij} \ket{e_i} \otimes \ket{f_j}$ and $g_2 = \sum_{i=1}^n \sum_{j=1}^m d_{ij} \ket{e_i} \otimes \ket{f_j}$

    \emph{To show the function has conjugate symmetric property:} \\
    $\braket{g_1 | g_2} = \sum_{i=1}^n \sum_{j=1}^m \overline{c_{ij}} d_{ij} = \sum_{i=1}^n \sum_{j=1}^m (c_{ij} \overline{d_{ij}})^\dagger = (\sum_{i=1}^n \sum_{j=1}^m c_{ij} \overline{d_{ij}} )^\dagger = \overline{\braket{g_2|g_1}} $

    \emph{To show the function has positive definite property:} \\
     $\braket{g_1|g_1} = \sum_{i=1}^n \sum_{j=1}^m \overline{c_{ij}} c_{ij} = \sum_{i=1}^n \sum_{j=1}^m |c_{ij}|^2 \geq 0$

    If $\braket{g_1|g_1} = 0 \implies \sum_{i=1}^n \sum_{j=1}^m |c_{ij}|^2 = 0 \implies |c_{ij}|^2 = 0 \text{ for all } 1 \leq i \leq n, 1 \leq j \leq m \implies c_{ij} = 0  \text{ for all } 1 \leq i \leq n, 1 \leq j \leq m \implies g_1 = 0_\mathcal{H}$

    Conversely, if $g_1 = 0_\mathcal{H} \implies c_{ij} = 0 \text{ for all } 1 \leq i \leq n, 1 \leq j \leq m \implies \braket{g_1|g_1} = 0$

    \emph{To show the function is conjugate linear in first variable:} \\

    Let $g_3 = \sum_{i = 1}^n \sum_{j=1}^m a_{ij} \ket{e_i} \otimes \ket{f_j}$

    $g_1 + g_2 = \sum_{i=1}^n \sum_{j=1}^m (c_{ij} + d_{ij}) \ket{e_i} \otimes \ket{f_j}$

    $\implies \braket{g_1 + g_2 | g_3} = \sum_{i=1}^n \sum_{j=1}^m [\overline{c_{ij} + d_{ij}}]a_{ij} = \sum_{i=1}^n \sum_{j=1}^m [\overline{c_{ij}} + \overline{d_{ij}}] a_{ij} = \sum_{i=1}^n \sum_{j=1}^m [\overline{c_{ij}} a_{ij}+ \overline{d_{ij}} a_{ij}] = \braket{g_1|g_3} \braket{g_2|g_3}$

    Also, $\braket{a \cdot g_1 | g_2} = \sum_{i=1}^n \sum_{j=1}^m \overline{a \cdot c_{ij}} d_{ij} = \sum_{i=1}^n \sum_{j=1}^m \overline{a} \; \overline{c_{ij}} d_{ij} = \overline{a}  \sum_{i=1}^n \sum_{j=1}^m \overline{c_{ij}} d_{ij} = \overline{a} \braket{g_1|g_2}$

    \emph{To show the function is linear in the second variable}

    $\braket{g_1| g_2 + g_3} = \braket{g_1|g_2} + \braket{g_1|g_3}$ can be shown similar to previous.

    Also, $\braket{ g_1 | a g_2} = \sum_{i=1}^n \sum_{j=1}^m \overline{c_{ij}} (a \cdot d_{ij}) = a \sum_{i=1}^n \sum_{j=1}^m \overline{c_{ij}} d_{ij} = a \braket{g_1|g_2}$

    $\implies \braket{g_1|g_2}$ is an inner product on $\mathcal{G}$
\end{proof}

\begin{defn}
    Consider Hilbert spaces $\mathcal{H}_1$ and $\mathcal{H}_2$.

    The vector space $\mathcal{G} =  \{ g: \mathcal{H}_1 \times \mathcal{H}_2 \to \mathbb{C} \; | \; g \text{ is conjugate linear and continuous } \}$  along with the inner product defined as previously is known as the \textbf{tensor product} of $\mathcal{H}_1$ and $\mathcal{H}_2$.
\end{defn}



\begin{samepage}
    \begin{mdframed}
\begin{lemma}[Principle of Entanglement]
When we have two qubits being treated as a combined system, the state space of the combined system is the tensor product $\mathcal{H}_1 \otimes \mathcal{H}_2$ of the state spaces $\mathcal{H}_1, \mathcal{H}_2$ of the component qubit subsystems. 

    Similarly, for a system of $n$ interacting qubits, the state space is the tensor product $\mathcal{H}_1 \otimes \mathcal{H}_2 \otimes ... \otimes \mathcal{H}_n$ of the state spaces of the $n$ qubits taken independently.
\end{lemma}
    \end{mdframed}
\end{samepage}

\begin{prop}
    Consider $\ket\psi \in \mathcal{H}_1, \ket\phi \in \mathcal{H}_2$.
    The linear functional $\ket{\psi} \otimes \ket{\phi}$ defined as $[\ket\psi \otimes \ket\phi](\ket\xi,\ket\eta) = \braket{\xi|\psi}_{\mathcal{H}_1}\braket{\eta|\phi}_{\mathcal{H}_2}$  satisfies the following properties:
    \begin{enumerate}
        \item $(a \ket\psi) \otimes \ket\phi = \ket\psi \otimes (a \ket\phi) = a (\ket\psi \otimes \ket\phi)$
        \item $a (\ket\psi \otimes \ket\phi) + b (\ket\psi \otimes \ket\phi) = (a + b) (\ket\psi \otimes \ket\phi)$
        \item $(\ket\psi_1 + \ket\psi_2) \otimes \ket\phi = \ket\psi_1 \otimes \ket\phi + \ket\psi_2 \otimes \ket\phi$
        \item $\ket\psi \otimes (\ket\phi_1 + \ket\phi_2) = \ket\psi \otimes \ket\phi_1 + \ket\psi \otimes \ket\phi_2$
    \end{enumerate}
\end{prop}

\begin{proof}
    [TODO]
\end{proof}

\begin{eg}
For $n=2$, the state space for $\mathcal{H} = \mathcal{H}_1 \otimes \mathcal{H}_2$ has computational basis $\{ \ket{00}, \ket{01}, \ket{10}, \ket{11} \} = \{ \ket{0}, \ket{1}, \ket{2}, \ket{3} \}$.

    Any arbitary state $\ket{\psi} \in \mathcal{H}$ can be described as $\ket{\psi} = a \ket{00} + b \ket{01} + c \ket{10} + d \ket{11} = a \ket{0} + b \ket{1} + c \ket{2} + d \ket{3}$ \\ where $a, b, c, d \in \mathbb{C}$ and $|a|^2 + |b|^2 + |c|^2 + |d|^2 = 1$.
\end{eg}


\begin{defn}
    A state $\ket{\psi} \in \mathcal{H}_1 \otimes \mathcal{H}_2 \otimes ... \otimes \mathcal{H}_n$ is said to be \textbf{entangled} if it cannot be written as a simple tensor product of states $\ket{v_1} \in \mathcal{H}_1, \ket{v_2} \in \mathcal{H}_2, ..., \ket{v_n} \in \mathcal{H}_n$. 

    If we can write $\ket{\psi} = \ket{v_1}\ket{v_2}...\ket{v_n} = \ket{v_1 v_2...v_n}$, the state is said to be \textbf{seperable}.
\end{defn}

\begin{eg}
The state $\ket{\psi_2} = \frac{1}{\sqrt{2}} (\ket{01} + \ket{11})$ of a $2$-qubit system is seperable. 

We can write $\ket{\psi_2} = \frac{1}{\sqrt{2}} (\ket{0} + \ket{1}) \otimes \ket{1}$
\end{eg}


\begin{eg}
The state $\ket{\psi} = \frac{1}{\sqrt{2}} (\ket{00} + \ket{11})$ of a $2$-qubit system is an entangled state.

    Assume that  $\ket{\psi} \frac{1}{\sqrt{2}} (\ket{00} + \ket{11}) $ can be decomposed as $\ket{\psi} = (\alpha_1 \ket{0}_1 + \beta_1 \ket{1}_1) \otimes (\alpha_2 \ket{0}_2 + \beta_2 \ket{1}_2) = \alpha_1 \alpha_2 \ket{00} + \alpha_1 \beta_2 \ket{01} + \beta_1 \alpha_2 \ket{10} + \beta_1 \beta_2 \ket{11}$.

    Equating the components, we find $\alpha_1 \alpha_2 = \displaystyle\frac{1}{\sqrt{2}}$, $\alpha_1 \beta_2 = 0$, $\beta_1 \alpha_2 = 0$ and $\beta_1 \beta_2 = \displaystyle\frac{1}{\sqrt{2}}$. These equations cannot be satisfied simulataneously as either one of $\alpha_1$ or $\beta_2$ has to be $0$.
\end{eg}


For Hilbert spaces $\mathcal{H}_1$ and $\mathcal{H}_2$ defining qubit systems, most states in the tensor product space $\mathcal{H}_1 \otimes \mathcal{H}_2$ of the interacting qubit systems are entangled.
\begin{prop}
    The set of seperable states has measure 0.
\end{prop}

\begin{proof}[intuition]
    refer https://physics.stackexchange.com/questions/268831/are-there-more-entangled-states-or-non-entangled-ones 

    Consider a state $\ket{\psi} = a \ket{0} + b \ket{1} \in \mathcal{H}_1$, $a,b \in \mathbb{C}$. Since $a$ and $b$ are complex coefficients, we would have $4$ degrees of freedom to assign a particular $\ket{\psi}$. However including the constraints that $a^2 + b^2 = 1$ and that multiplying by global phase leaves the state unchanged, we are effectively left with $2$ degrees of freedom for assigning $\ket{\psi}$. 
    
Similarly assigning $\ket{\phi} \in \mathcal{H}_2$ has $2$ degrees of freedom.

    Consider the $4$-dimensional tensor space $\mathcal{H}_1 \otimes \mathcal{H}_2$. Since the state of any vector $\ket{\omega}$ in this space can be written as $\ket{\omega} = a \ket{00} + b \ket{01} + c \ket{10} + d \ket{11}$, where $a, b, c, d \in \mathbb{C}$ we have $8$ degrees of freedom initially for assigning the vector $\ket{\omega}$. Including constraint $a^2 + b^2 + c^2 + d^2 = 1$ and that multiplying by global phase leaves the state unchanged, we have $6$ degrees of freedom in assigning the value of $\ket{\omega}$ which is $2$ degrees of freedom more than $4 = 2 \times 2$ from the individual qubits.
\end{proof}
