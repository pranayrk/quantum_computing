\section{Measurement}
The principle of superposition might indicate that we can use the continuum state of single qubit to store an infinite amount of information. However, a principal of quantum mechanics states that we cannot interact with the qubit without fundamentally altering its state. To know the state stored in a qubit, we must perform a measurement which forces the state of the qubit to "collapse" into one of two \textit{preferred states}.

A naive version principle of measurement for a single qubit is stated below. We will formalize this notion and generalize it to multiple qubits.
\begin{lemma}[Principle of Measurement]
    Any measurement device that interacts with the qubit will be calibrated with a pair of orthonormal vectors called the \textbf{preferred basis}, say $\{ \ket{u}, \ket{v} \}$. If the state of the qubit with respect to the preferred basis is $\ket{\psi} = a \ket{u} + b \ket{v}$, then measurement of the qubit will yield either $\ket{u}$ with a probability of $|a|^2$ or $\ket{v}$ with a probability $|b|^2$. \\
The process of measurement leads to the quantum state vector $\ket{\psi}$ undergoing a discontinuous change which leads to the collapse of the state vector onto one of the vectors in the preferred basis.
\end{lemma}

To formalize this notion, we have two main options: projection-valued measures (PVM) and positive-operator-valued measure (POVM). We will proceed to describe PVMs here.

\begin{defn}
    An \textbf{observable} is a physically measurable quantity of a quantum system which is represented by a self-adjoint operator on the Hilbert space associated with the quantum system.
\end{defn}

TODO: Add direct product in dirac notation


\begin{lemma}
    The eigenvectors of an observable form an orthonormal basis for the Hilbert space.
\end{lemma}

\begin{lemma}
    In a qubit represented by Hilbert space $\mathcal{H}$, the possible measurement values of an observable are given by the spectrum $\sigma(A)$ of the self adjoint operator $A$ representing the observable.

    The probability $p_\psi(\lambda)$ that a quantum system in the pure state $\ket{\psi} \in \mathcal{H}$ yields the eigenvalue $\lambda$ of $A$ upon measurement is given by the projection $P_\lambda$ onto the eigenspace $\text{Eig}(A, \lambda)$ of $\lambda$ as $p_\psi(\lambda) = || P_\lambda \ket{\psi} ||^2$
\end{lemma}

\begin{lemma}[Principle of Measurement]
Any physical observable is associated with a self-adjoint operator $\mathcal{A}$ on the Hilbert space $\mathcal{H}_S$. The possible outcome of a measurement of the observable $\mathcal{A}$ is one of the eigenvalues of the operator $\mathcal{A}$. \\
Writing the eigenvalues equation, $\mathcal{A} \ket{i} = a_i \ket{i}$ where $\ket{i}$ is an orthonormal basis of eigenvectors of the operator $\mathcal{A}$, and  $\ket{\psi} = \sum_i c_i \ket{i}$,  then the probability that a measurement of the observable $\mathcal{A}$ results in the outcome $a_i$ is given by $p_i = |\braket{i|\psi}|^2 = |c_i|^2$
\end{lemma}


\begin{defn}
    A \textbf{density operator} is a positive semi-definite operator on the Hilbert space whose trace is equal to 1.
\end{defn}

\begin{lemma}
    For each measurement that can be defined, the probability distribution over the outcomes of the measurement can be computed from the density operator as defined by Born's rule:
    $P(x_i) = \text{tr}(\Pi_i \rho)$ where $\rho$ is the density operator and $\Pi_i$ is the projection operator onto the baiss vector corresponding to the measurement outcome $x_i$.
\end{lemma}

\begin{lemma}
    The expectation value of a quantum state $\rho$ is $<A> = \text{tr}(A \rho)$.
\end{lemma}

\begin{defn}
    Let $\mathcal{H}$ be a Hilbert space. We call states $\ket{\psi_1}, \ket{\psi_2}, ..., \ket{\psi_n} \in \mathcal{H}$ perfectly distinguishable if there exists a measurement system $\{ M_i \}_{i=1}^m$ with $m \geq n$ such that $|| M_j \ket{\psi_1} ||^2 = \begin{cases} 1 & \text{if } i = j \\ 0 & \text{if } i \neq j \end{cases}$
\end{defn}
Here \textit{perfectly distinguishable} means that there is some experiment or experimental setup that can distinguish between these two states, atleast in theory.
\begin{result}
    The states $\ket{\psi_1}, \ket{\psi_2}, ..., \ket{\psi_n}$ are perfectly distinguishable if and only if they are orthogonal. This result is the reason we use orthogonal basis in quantum computing.
\end{result}
`TODO: Refer Nielsen, Chuang

This property limits the amount of information that can be extracted from a qubit: a measurment yields atmost a single classical bit worth of information. In most cases, we also cannot make more than one measurement of original state of the qubit. On measurement, we have two possibilities, each corresponding to a probability of $|a|^2$ and $|b|^2$, then the total probability of the whole space will be $|a|^2 + |b|^2 = 1$, which is valid for unit vectors $\ket{\psi} = a \ket{0} + b \ket{1}$.

\begin{notation}
    When $\ket{\psi} = a \ket{0} + b \ket{1} = \begin{bmatrix} a \\ b \end{bmatrix}$, then $\bra{\psi}$ is the conjugate transpose of $\ket{\psi}$ and is read as \textbf{bra psi}, $\bra{\psi} = \begin{bmatrix} \overline{a} & \overline{b}\end{bmatrix}$
\end{notation}

This lets us write the inner product for $\mathcal{H}$ as:
For any $\ket{v} = \begin{bmatrix} a \\ b \end{bmatrix}, \ket{w} = \begin{bmatrix} c \\ d \end{bmatrix} \in \mathcal{H}$, the operation $\braket{v|w} = \bra{v}\ket{w} = \begin{bmatrix} \overline{a} & \overline{b} \end{bmatrix} \begin{bmatrix} c \\ d \end{bmatrix} = \overline{a} c + \overline{b} d $

We will consider the inner product as being linear in the second variable and conjugate-linear in the first variable.

\begin{remark}
If $\ket{\psi} = \begin{bmatrix} a \\ b \end{bmatrix}$, then we can show $\braket{0|\psi} = a$, $\braket{1|\psi} = b$.
Therefore we can write $\ket{\psi} = a \ket{0} + b \ket{1} = \braket{0|\psi} \ket{0} + \braket{1|\psi} \ket{1}$.
\end{remark}

\begin{remark}
The standard inner product of the $\ket{\psi} = \begin{bmatrix} a \\ b \end{bmatrix}$ with itself in the Hilbert space $\mathcal{H}$ can therefore be written as $\braket{\psi | \psi} = \bra{\psi}\ket{\psi} = \begin{bmatrix} \overline{a} & \overline{b}\end{bmatrix} \begin{bmatrix} a \\ b \end{bmatrix} = |a|^2 + |b|^2 = 1$
\end{remark}

`TODO: Proof that self-adjoint matrices represent measurement operators`
`TODO: Relation of POVM and matrices`

Let $\mathcal{H}_1$ be an $n$-dimensional vector space with basis $\alpha = \{ \ket{a_1}, \ket{a_2}, ..., \ket{a_n} \}$ and $\mathcal{H}_2$ be an $m$-dimensional vector space with basis $\beta = \{ \ket{b_1}, \ket{b_2}, ..., \ket{b_n} \}$, then the tensor product $\mathcal{H}_1 \otimes \mathcal{H}_2$ is an $nm$-dimensional space with basis elements of the form $\ket{a_i} \otimes \ket{b_j}$

\begin{notation}
    In dirac's bra/ket notation, the tensor product of $\ket{v} \in \mathcal{H}_2, \ket{w} \in \mathcal{H}_2$ is $\ket{vw} = \ket{v}\ket{w} = \ket{v} \otimes \ket{w}$
\end{notation}

The tensor product is defined to satisfy the following properties:
\begin{enumerate}
    \item $(\ket{v_1} + \ket{v_2} ) \ket{w} = \ket{v_1}\ket{w} + \ket{v_2}\ket{w}$
    \item $\ket{v}(\ket{w_1} + \ket{w_2}) =  \ket{v}\ket{w_1} + \ket{v}\ket{w_2}$
    \item $(a \cdot \ket{v})\ket{w} = \ket{v}(a \cdot \ket{w})  = a \cdot (\ket{v}\ket{w})$
\end{enumerate}
Every element $\ket{\sigma} \in \mathcal{H}_1 \otimes \mathcal{H}_2$ can be written as a superposition of elements of the basis $\{ \ket{a_i}\ket{bj} \}$ as $\ket{\sigma} = \alpha_{11} \ket{a_1 b_1} + \alpha_{12} \ket{a_1 b2} + ... + \alpha_{nm} \ket{a_n b_m}$.

Most elements $\ket{\sigma} \in \mathcal{H}_1 \otimes \mathcal{H}_2$ \textit{cannot} be decomposed to $\ket{\sigma} = \ket{v}\ket{w}$ where $v \in \mathcal{H}_1, w \in \mathcal{H}_2$.
`TODO: Check proof`

Here \textit{perfectly distinguishable} means that there is some experiment or experimental setup that can distinguish between these two states, atleast in theory.

\begin{defn}
            Let $\mathcal{H}$ be a Hilbert space. We call states $\ket{\psi_1}, \ket{\psi_2}, ..., \ket{\psi_n} \in \mathcal{H}$ perfectly distinguishable if there exists a measurement system $\{ M_i \}_{i=1}^m$ with $m \geq n$ such that $|| M_j \ket{\psi_1} ||^2 = \begin{cases} 1 & \text{if } i = j \\ 0 & \text{if } i \neq j \end{cases}$
\end{defn}

\begin{result}
            The states $\ket{\psi_1}, \ket{\psi_2}, ..., \ket{\psi_n}$ are perfectly distinguishable if and only if they are orthogonal. This result is the reason we use orthogonal basis in quantum computing.
\end{result}

Positive-Operator-Valued Measures (POVMs) are a further generalization of the Projection-Valued Measure (PVMs) and are described in the appendix. 
