\section{Measurement}
The principle of superposition might indicate that we can use the continuum state of single qubit to store an infinite amount of information. However, a principal of quantum mechanics states that we cannot interact with the qubit without fundamentally altering its state. To know the state stored in a qubit, we must perform a measurement which forces the state of the qubit to "collapse" into one of two \textit{preferred states}.

A naive version principle of measurement for a single qubit is stated below. We will develop a further formalism for this notion and generalize it to multiple qubits.
\begin{samepage}
\begin{mdframed}
\begin{axiom}[Principle of Measurement]
    Any measurement device that interacts with the qubit will be calibrated with a pair of orthonormal vectors called the \textbf{preferred basis}, say $\{ \ket{u}, \ket{v} \}$. If the state of the qubit with respect to the preferred basis is $\ket{\psi} = a \ket{u} + b \ket{v}$, then measurement of the qubit will yield either $\ket{u}$ with a probability of $|a|^2$ or $\ket{v}$ with a probability $|b|^2$. \\
The process of measurement leads to the quantum state vector $\ket{\psi}$ undergoing a discontinuous change which leads to the collapse of the state vector onto one of the vectors in the preferred basis.
\end{axiom}
\end{mdframed}
\end{samepage}

\begin{defn}
    For a linear map $A: \mathcal{H}_1 \to \mathcal{H}_2$, the \textbf{adjoint} $A^*$ is defined to satisfy the relation 
    $\braket{A \psi| \phi} = \braket{\psi | A^* \phi}$ for all $\ket\psi \in \mathcal{H}_1,\ket\phi \in \mathcal{H}_2$.

    A matrix $A: \mathcal{H} \to \mathcal{H}$ which is its own adjoint is said to be \textbf{self-adjoint} (or \textbf{Hermitian} in the finite dimensional case).

    A Hermitian operator $A$ will satisfy $\braket{A \psi | \phi} = \braket{\psi | A \phi}$ for all $\ket\psi, \ket\phi\in \mathcal{H}$.
\end{defn}

\begin{prop}
    The eigenvalues of a Hermitian operator are real.
\end{prop}
\begin{proof}
    Let $\lambda$ be an eigenvalue of $A$ and $\ket\psi \in \mathcal{H}$ be its associated eigenvector, i.e. $A \ket\psi = \lambda \ket\psi$ for some $\lambda \in \mathbb{C}$.

    $\lambda\braket{\psi|\psi} = \braket{\lambda \psi|\psi} = \braket{\psi|\lambda \psi} = \overline{\braket{\lambda \psi | \psi}} = \overline{\lambda} \braket{\psi|\psi} \implies \lambda = \overline{\lambda} \implies \lambda $ is real.

\end{proof}

\begin{prop}
    The eigenvectors of a Hermitian operator are orthogonal.
\end{prop}
\begin{proof}
    Given a Hermitian matrix $A$, consider two distinct eigenvalues $\lambda_1$ and $\lambda_2$.

    Let $\ket{\psi_1}$ be the eigenvector corresponding to $\lambda_1$ and $\ket{\psi_2}$ be the eigenvector corresponding to $\lambda_2$.

    Then $A \ket{\psi_1} = \lambda_1 \ket{\psi_1}$ and  $A \ket{\psi_2} = \lambda_2 \ket{\psi_2}$.

    Then $\braket{\psi_1 \; | \; A \psi_2} = \bra{\psi_1} A \ket{ \psi_2} = \bra{\psi_1} \lambda_2 \ket{\psi_2} = \lambda_2 \bra{\psi_1} \ket{\psi_2}$ (since $\lambda_2$ is real $\overline{\lambda_2} = \lambda_2$).
    
    Also, $\braket{\psi_1 \; | \; A \psi_2} = \braket{A \psi_1 \; | \; \psi_2} = \bra{\psi_1} \overline{A} \ket{\psi_2} = \bra{\psi_1} \lambda_1 \ket{\psi_2} = \lambda_1 \braket{\psi_1|\psi_2}$
    

    Subtracting the above equations, we have $(\lambda_1 - \lambda_2) \braket{\psi_1|\psi_2} = 0$.
    Since $\lambda_1$ and $\lambda_2$ are distinct eigenvalues $(\lambda_1 - \lambda_2) \neq 0 \implies \braket{\psi_1|\psi_2} = 0 \implies \ket{\psi_1} \text{ and } \ket{\psi_2} $ are orthogonal.

\end{proof}

\begin{defn}
    An \textbf{observable} is a physically measurable quantity of a quantum system which is represented by a self-adjoint operator on the Hilbert space associated with the quantum system.
\end{defn}

\begin{lemma}
    The eigenvectors of an observable form an orthonormal basis for the Hilbert space.
\end{lemma}

\begin{lemma}
    In a qubit represented by Hilbert space $\mathcal{H}$, the possible measurement values of an observable are given by the spectrum $\sigma(A)$ of the self adjoint operator $A$ representing the observable.

    The probability $p_\psi(\lambda)$ that a quantum system in the pure state $\ket{\psi} \in \mathcal{H}$ yields the eigenvalue $\lambda$ of $A$ upon measurement is given by the projection $P_\lambda$ onto the eigenspace $\text{Eig}(A, \lambda)$ of $\lambda$ as $p_\psi(\lambda) = || P_\lambda \ket{\psi} ||^2$
\end{lemma}

The more general Hermitian operator formalism for the measurement principle is stated below.

\begin{mdframed}
\begin{axiom}[Principle of Measurement]
Any physical observable is associated with a self-adjoint operator $A$ on the Hilbert space $\mathcal{H}_S$. 

The possible outcome of a measurement of the observable $A$ is one of the eigenvalues of the operator $\mathcal{A}$.

Writing the eigenvalues equation, $A \ket{\psi_i} = \lambda_i \ket{\psi_i}$ where $\ket{\psi_i}$ is an orthonormal basis of eigenvectors of the operator $A$, and  $\ket{\psi} = \sum_i \lambda_i \ket{\psi_i}$,  then the probability that a measurement of the observable $A$ results in the outcome $\lambda_i$ is given by $P_i = |\braket{\psi_i|\psi}|^2 = |\lambda_i|^2$

If the eigenvalue $\lambda_i$ was measured, then the quantum state after measurement is a unit length eigenvector of $A$ corresponding to eigenvalue $\lambda_i$.
\end{axiom}
\end{mdframed}


This property limits the amount of information that can be extracted from a qubit: a measurment yields atmost a single classical bit worth of information. In most cases, we also cannot make more than one measurement of original state of the qubit. On measurement, we have two possibilities, each corresponding to a probability of $|a|^2$ and $|b|^2$, then the total probability of the whole space will be $|a|^2 + |b|^2 = 1$, which is valid for unit vectors $\ket{\psi} = a \ket{0} + b \ket{1}$.
