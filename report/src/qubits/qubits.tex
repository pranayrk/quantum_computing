\chapter{Qubits}

\line(1,0){360} \\

The computers we use today are based on \textbf{bits} (binary digits) each of which can represent a $0$ or $1$ state. The rules governing these bits are laid out in classical information theory and these computers can be considered equivalent to a ideal abstract computational framework - the Turing Machine.

By exploiting certain phenomena observed in the working of quantum particles, we can derive a model of a computer which can achieve results that can not be replicated efficiently on a Turing Machine. In these quantum computers, the \textbf{qubit} (quantum bit) forms the foundational unit of computing.

Many different quantum particle effects have been used in labs - photon polarization, electron spin, the state of an atom in a cavity, and even defect centers in a diamond have been leveraged to created real life implementations of qubits. We will define a qubit as a mathematical object with a certain ruleset and expect that every real-world implementation follows the working of the abstract model.

\begin{samepage}
\begin{defn}
    A complex \textbf{Hilbert space} $\mathcal{H}$ is a vector space over $\mathbb{C}$ with a positive definite inner product $\braket{} : \mathcal{H} \times \mathcal{H} \to \mathbb{C}$ defined as $(\psi,\phi) \to \braket{\psi,\phi}$ such that for all $\phi, \phi_1, \phi_2, \psi \in \mathcal{H}$ and $a, b \in \mathbb{C}$ the inner product is:
    \begin{enumerate}
        \item conjugate symmetric: $\braket{\psi,\phi} = \overline{\braket{\phi,\psi}}$
        \item positive definite: $\braket{\psi, \psi} \geq 0$ and $\braket{\psi, \psi} = 0 \iff \psi = 0$
        \item conjugate-linear in first argument: $\braket{a \phi_1 + b \phi_2, \psi} = \overline{a} \braket{\phi_1, \psi} + \overline{b} \braket{\phi_2, \psi}$
        \item linear in second argument: $\braket{\psi, a \phi_1 + b \phi_2} = a \braket{\psi,\phi_1} + b \braket{\psi,\phi_2}$
    \end{enumerate}
    and this inner product induces a norm $||.|| : \mathcal{H} \to \mathbb{R}$ defined as $\psi \to \sqrt{\braket{\psi,\psi}}$ in which $\mathcal{H}$ is complete.
\end{defn}

\begin{note}
    We have set the inner product to be linear in the second argument and anti-linear in the first argument. This is to make later calculations easier.
\end{note}
\end{samepage}

\begin{defn}
    Given a matrix $A$, the \textbf{conjugate transpose} $A^\dagger$ is obtained by transposing $A$ and applying the complex conjugate of each entry.

    $A^\dagger = (\overline{A})^T = \overline{(A^T)}$ where $\overline{A}$ is the complex conjugate of $A$ and $A^T$ is the transpose of $A$.
\end{defn}

\begin{result}
    The conjugate transpose has the following properties:
    \begin{itemize}
        \item $(A + B)^\dagger = A^\dagger + B^\dagger$
        \item $(c \cdot A)^\dagger = \overline{c} A^\dagger$ for any $c \in \mathbb{C}$
        \item $(AB)^\dagger = B^\dagger A^\dagger$
        \item $(A^\dagger)^\dagger = A$
    \end{itemize}
\end{result}


\begin{samepage}
\begin{defn}
    A \textbf{qubit} is any quantum mechanical system whose state can be completely described by a unit vector in a $2$-dimensional complex Hilbert space $\mathcal{H}$ and which follows these axioms: 
\begin{itemize}
    \item Principle of Superposition
    \item Principle of Entanglement
    \item Principle of Measurement
    \item Principle of Transformation
\end{itemize}
    The Hilbert space $\mathcal{H}$ is known as the \textbf{state space} and is equipped with the inner product $\braket{}$ which is defined as $\braket{\psi, \phi} = \begin{bmatrix} a \\ b \end{bmatrix}^\dagger \begin{bmatrix} c \\ d \end{bmatrix} = \begin{bmatrix} \overline{a} & \overline{b} \end{bmatrix} \begin{bmatrix} c \\ d \end{bmatrix} = \overline{a} c + \overline{b} d$ for any $\psi = \begin{bmatrix} a\\b\end{bmatrix}, \phi = \begin{bmatrix} c \\ d \end{bmatrix} \in \mathcal{H}$. Any unit vector of $\mathcal{H}$ is called a \textbf{state vector}.
\end{defn}

The principles in the above definition will be elaborated on in the upcoming sections.
They are empirical observations of the behaviour of quantum mechanical systems and will be considered as axioms in our abstract qubit system.
\end{samepage}


\begin{defn}
    Any function $\phi: V \to \mathbb{F}$ from a vector space to its base field is called a \textbf{functional}. 
\end{defn}

\begin{defn}
A linear functional $\phi$ on a normed linear space $V$ is said to be \textbf{bounded} if there exists some real $M$ such that $|| \phi(v)|| \leq M ||v||$ for all $v \in V$. 
\end{defn}

\begin{result}
A linear functional $\phi$ being bounded is equivalent to $\phi$ being continuous.
\end{result}

\begin{defn}
    The set of all continuous linear functionals on a vector space $V$ is known as the \textbf{continuous dual space} of $V$.
\end{defn}

\begin{result}[Riesz' representation theorem]
    For any continuous linear functional $\phi$ on a Hilbert space $\mathcal{H}$, there exists a unique $u \in \mathcal{H}$ such that $\phi(v) = \braket{u,v}$ for all $v \in \mathcal{H}$. 
\end{result}

In fact the converse is true as well.

\begin{prop}
\label{innerproduct:continuous}
For a fixed $\psi \in \mathcal{H}$, consider a linear functional $f_\psi : \mathcal{H} \to \mathbb{C}$ such that $f_\psi(\phi) = \braket{\psi,\phi}$ for all $\phi \in \mathcal{H}$. Then $f_\psi$ is continuous and unique. 
\end{prop}
\begin{proof}
    \textit{To verify $f_\psi$ is continuous:} \\
    $f_\psi$ is continuous if and only if it is bounded. 

    The Cauchy-Schwarz inequality for inner product tells us that $|\braket{\psi,\phi}| \leq ||\psi||||\phi||$.

    This implies $|f_\psi| = |\braket{\psi,\phi}| \leq ||\psi|| ||\phi|| = M ||\phi||$ where $M = ||\psi||$ is a fixed quantity for a fixed $\psi \in \mathcal{H}$, i.e. $|f_\psi|$ is bounded.

    Hence $f_\psi$ is continuous. 

    \textit{To verify $f_\psi$ is unique:} \\
    
    Since $f_\psi$ is a continuous linear functional, it is an element of the continuous dual space of $\mathcal{H}$ and is therefore unique by Riesz's representation theorem.
\end{proof}

The above properties of Hilbert spaces justifies the \textbf{Dirac Bra/Ket Notation} which is widely used in quantum mechanics, and which we will follow in this paper.

\begin{note}
    The \textbf{Dirac Bra/Ket Notation} is ubiquitously used in quantum mechanics and quantum information science. It allows vectors, continuous dual functions, and inner products to be represented conveniently. 

    Any vector $\psi \in \mathcal{H}$ will be written as $\ket{\psi}$ and is read \textit{ket psi}.

    The unique continuous linear functional $f_\psi$ defined as $f_{\psi}(\phi) = \braket{\psi, \phi}$ for all $\phi \in \mathcal{H}$ and a fixed $\psi \in \mathcal{H}$ will be written as $\bra{\psi}$ and is read \textit{bra psi}.

    The inner product $\braket{\psi, \phi}$ will then be written as $\bra{\psi} \ket{\phi}$ or more simply as $\braket{\psi|\phi}$.
\end{note}

\begin{prop}
    For any $\ket{\psi} = \begin{bmatrix} a \\ b\end{bmatrix} \in \mathcal{H}$, the linear functional $\bra{\psi}$ has the matrix representation $\bra\psi = \ket{\psi}^\dagger = \begin{bmatrix} \overline{a} & \overline{b} \end{bmatrix}$. 
\end{prop}
\begin{proof}

    Consider $\ket{\psi} = \begin{bmatrix} a \\ b\end{bmatrix}, \ket{\phi} = \begin{bmatrix} c \\ d \end{bmatrix} \in \mathcal{H}$.

        Then $\ket{\psi}^\dagger \ket{\phi} = \begin{bmatrix}\overline{a} & \overline{b} \end{bmatrix}  \begin{bmatrix} c \\ d \end{bmatrix} = \overline{a} c + \overline{b} d = \braket{\psi|\phi} = f_{\ket{\psi}}(\ket{\phi})$ for some continuous linear functional $f_{\ket{\psi}}: \mathcal{H} \to \mathbb{C}$.

            Since $f_{\ket\psi}$ is unique by Riesz's representation theorem, we can set $\bra{\psi} = f_{\ket\psi}(\phi) = \ket{\psi}^\dagger \ket{\phi}$ for all $\ket\phi \in \mathcal{H}$, i.e. $\bra{\psi}$ is the linear functional which has matrix representation $\ket{\psi}^\dagger$.
\end{proof}
