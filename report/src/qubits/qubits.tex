\chapter{Qubits}

\line(1,0){360} \\

The computers we use today rely on classical information theory, which are based on \textbf{bits} (binary digits) which can represents a $0$ or $1$ state. These \textbf{classical computers}) are equivalent to a Turing Machine in computational efficiency. 

On a quantum computer the \textbf{qubit} (quantum bit) is the basic unit of information.

In labs, qubits have been implemented using photon polarization, electron spin, the ground/excited state of an atom in a cavity, and even defect centers in a diamond. In this paper, we will define a qubit as an abstracted mathematical object and expect that every real-world implementation follow the same rules.

\begin{defn}
    A \textbf{Hilbert space} $\mathcal{H}$ is a complete vector space with a positive definite scalar product $\braket{.|.} : \mathcal{H} \times \mathcal{H} \to \mathbb{C}$ defined as $(\psi,\phi) \to \braket{\psi|\phi}$ such that for all $\phi, \phi_1, \phi_2, \psi \in \mathcal{H}$ and $a, b \in \mathbb{C}$:
    \begin{enumerate}
        \item $\braket{\psi | \phi} = \overline{\braket{\phi | \psi}}$
        \item $\braket{\psi | \psi} \geq 0$
        \item $\braket{\psi | \psi} = 0 \iff \psi = 0$
        \item $\braket{\psi | a \phi_1 + b \phi_2} = a \braket{\psi | \phi_1} + b \braket{\psi | \phi_2}$
    \end{enumerate}
    and this scalar product induces a norm $||.|| : \mathcal{H} \to \mathbb{R}$ defined as $\psi \to \sqrt{\braket{\psi|\psi}}$ in which $\mathcal{H}$ is complete.
\end{defn}

\begin{samepage}
\begin{defn}
A \textbf{qubit} is any quantum mechanical system which is associated with $2$-dimensional complex Hilbert space $\mathcal{H}$ known as the \textbf{state space} and follows the below principles:
\begin{itemize}
    \item Principle of Superposition
    \item Principle of Entanglement
    \item Principle of Measurement
    \item Principle of Transformation
\end{itemize}
 A given state of the system is completely described by a
 \textit{unit vector} $\ket{\psi}$, which is called the \textbf{state vector} (or wave function) on the Hilbert Space
\end{defn}

The principles in the above definition will be elaborated on in the upcoming sections.

\end{samepage}

\begin{notation}
    Observe above that we have written the vector $\vec{\psi} \in \mathcal{H}$ as $\ket{\psi}$. This is the notation for a vector in \textnormal{Dirac's bra/ket notation}, and is read \textbf{ket psi}
\end{notation}
