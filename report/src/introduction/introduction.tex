\chapter{A brief history of Quantum Computing}
\line(1,0){360} \\ \\
In the last decades of the twentieth century, certain scientists sought to combine two recent theories that were highly influential: \textbf{Information Theory} and \textbf{Quantum Mechanics}.

\begin{itemize}
\item \textbf{1984:} Charles Bennet and Gilles Brassad published a quantum key distribution protocol now called \textbf{BB84}, allowing two parties to establish an absolutely secure secret key.
\item \textbf{1980s:} Feynman recognized that a system of
    $n$-particle quantum systems could not be simulated efficiently by a Turing machine, seemingly requiring time/space that is exponential in $n$. He proposed that computers based on quantum systems could simulate quantum processes with more efficiency. This led to the question: If simluating quantum problems was more efficient on a quantum computer, would there be other problems that would run more efficiently on a quantum computer?
\item \textbf{1994:} Peter Shor found the famous \textbf{Shor's Algorithm} for a quantum computer which could factors $n$-digit integers into primes with $n^2$ efficiency (with high probability). The fastest known algorithm for factoring $n$-digit integers in classical computing is of efficiency around $2^{n^{1/3}}$
\end{itemize}
