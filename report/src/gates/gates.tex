\chapter{Gates and Circuits}
\line(1,0){360} \\

\begin{comment}
     x Definitition of quantum gates
     x Gates are unitary
     x reversible defn
     x Gates are reversible
     x quantum circuit definition
     * Quantum Gates applied to one qubit in an $n$-qubit register
     * deferred measurement principle
     * ket bra notation and how it describes a transformation
     x any reversible classical gate has a valid quantum analog
     x Quantum gates have an equal number of inputs and outputs?
     * Single Qubit Gates
     * Eigen values and Eigenvectors of Hadamard Gate
     * Multiple Qubit Gates
     * Proof that each gate described is unitary
\end{comment}


\begin{defn}
    A \textbf{quantum gate} is a function $U: \mathcal{H} \to \mathcal{H}$ such that $f(\ket\psi) = \ket\phi$ where $\ket\psi, \ket\phi \in \mathcal{H}$ are valid quantum states in the state space $\mathcal{H}$ of $n$ interacting qubits.
\end{defn}

\begin{prop}
    Any quantum gate is unitary and reversible.
\end{prop}
\begin{proof}
    The principle of transformation (Lemma \ref{transformation}) tells us that the time-evolution of the quantum state is linear and unitary. 

    This implies any quantum gate $U$ is unitary, i.e. $U U^{\dagger} = I$. 

    $U U^\dagger = I \implies U^{-1} = U^\dagger$.
    Also, $ U U^\dagger = I \implies U^\dagger$ is unitary.  
    This means reverse the operation of any quantum gate $U$ on the system by applying the quantum gate $U^\dagger$.
\end{proof}

\begin{defn}
    A \textbf{quantum circuit} is a sequence of quantum gates and measurement operators applied to an $n$-qubit register initialized to some known quantum state.
\end{defn}

\begin{prop}[Deferred Measurement Principle]
    Every quantum circuit is equivalent to a circuit in which all measurements are made after all other computations.
\end{prop}

