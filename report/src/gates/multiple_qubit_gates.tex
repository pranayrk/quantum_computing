\section{Gates on Multiple Qubits}

\subsection{CNOT Gate}
\begin{defn}
    The \textbf{CNOT gate} is a gate that acts on 2 qubits which flips the second bit if the first bit is in the $\ket{1}$ state.\\
    It is defined by the matrix $\text{CNOT} = \begin{bmatrix} 1 & 0 & 0 & 0 \\ 0 & 1 & 0 & 0 \\ 0 & 0 & 0 & 1 \\ 0 & 0 & 1 & 0 \end{bmatrix} = \ket{0}\bra{0} \otimes I + \ket{1}\bra{1} \otimes X$
\end{defn}
The CNOT gate allows us to obtain an entangled state.

\subsection{Toffoli Gate}
\begin{defn}
    The \textbf{Toffoli gate} is a gate that acts on 3 qubits that flips the third bit if the first two are in the $\ket{1}$ state.\\
\end{defn}

Depending on the input the Toffoli gate can function as an AND, NOT and NAND gate. Since the NAND gate is universal, the Toffoli is as well. The Toffoli gate is also unitary which means it is a valid quantum gate. This shows that every classical circuit can be implemented as a quantum circuit.

\subsection{Hadamard Transform}
\begin{defn}
    Given a register of $n$ qubits, the \textbf{Hadamard Transform} $H^{\otimes n}$ is the transformation that applies the Hadamard gate $H = \begin{bmatrix} 1 & 1 \\ 1 & -1 \end{bmatrix}$ on each of the $n$ qubits.
\end{defn}
\begin{eg}
    Consider the Hadamard transform applied on a $n$ qubit register, where each qubit is in the $\ket{0}$ state, i.e. the register is in the state $\ket{0^n}$.
    Then \[ H^{\otimes n} \ket{0^n} = \frac{1}{2^{n/2}} (\ket{0} + \ket{1})(\ket{0} + \ket{1})...(\ket{0} \ket{1}) = \frac{1}{2^{n/2}} \sum_{j=0}^{2^n-1} \ket{j} \] where $\ket{j}$ is the bitstring that represents $j$ in binary.
\end{eg}

\begin{result}
    For any arbitrary state $\ket{j}$ in an $n$ qubit register $\otimes_{i=1}^n \mathcal{H}_i$, \[ H^{\otimes n} \ket{j} = \frac{1}{2^{n/2}} \sum_{k=0}^{2^n-1} (-1)^{j.k} \ket{k}\] where $j.k$ is the dot product of $j$ and $k$.
\end{result}
`Refer Shor Lec 16`
