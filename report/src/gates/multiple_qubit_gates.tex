\section{Gates on Multiple Qubits}

\subsection{CNOT Gate}
\begin{defn}
    The \textbf{CNOT gate} is a gate that acts on 2 qubits which flips the second bit if the first bit is in the $\ket{1}$ state.\\
    It is defined by the matrix $\text{CNOT} = \begin{bmatrix} 1 & 0 & 0 & 0 \\ 0 & 1 & 0 & 0 \\ 0 & 0 & 0 & 1 \\ 0 & 0 & 1 & 0 \end{bmatrix} = \ket{0}\bra{0} \otimes I + \ket{1}\bra{1} \otimes X = \ket{00}\bra{00} + \ket{01}\bra{01} + \ket{11} \bra{10} + \ket{10} \bra{11}$ 
\end{defn}
The CNOT gate allows us to obtain an entangled state.
\begin{eg}
    The state $\frac{1}{\sqrt{2}}(\ket{00} + \ket{10})$ is seperable.

    $\text{CNOT} \frac{1}{\sqrt{2}}(\ket{00} + \ket{10}) = \frac{1}{\sqrt{2}} (\ket{00} + \ket{11}) $ which is an entangled state.
\end{eg}

The CNOT gate is its own inverse. This means it can also take an entangled state to a seperable one.

\subsection{Hadamard Transform}
\begin{defn}
    Given a register of $n$ qubits, the \textbf{Hadamard Transform} $H^{\otimes n}$ is the transformation that applies the Hadamard gate $H = \begin{bmatrix} 1 & 1 \\ 1 & -1 \end{bmatrix}$ on each of the $n$ qubits.
\end{defn}
\begin{eg}
    Consider the Hadamard transform applied on a $n$ qubit register, where each qubit is in the $\ket{0}$ state, i.e. the register is in the state $\ket{0^n}$.
    Then \[ H^{\otimes n} \ket{0^n} = \frac{1}{2^{n/2}} (\ket{0} + \ket{1})(\ket{0} + \ket{1})...(\ket{0} \ket{1}) = \frac{1}{2^{n/2}} \sum_{j=0}^{2^n-1} \ket{j} \] where $\ket{j}$ is the bitstring that represents $j$ in binary.
\end{eg}

\begin{result}
    For any arbitrary state $\ket{j}$ in an $n$ qubit registe , \[ H^{\otimes n} \ket{j} = \frac{1}{2^{n/2}} \sum_{k=0}^{2^n-1} (-1)^{j.k} \ket{k}\] where $j.k$ is the dot product of the bitstrings $j$ and $k$.
\end{result}

\begin{eg}
    $H^{\otimes 5} \ket{01011} = \frac{1}{\sqrt{32}} (\ket{0} + \ket{1}) (\ket{0} - \ket{1}) (\ket{0} + \ket{1}) (\ket{0} - \ket{1}) (\ket{0} - \ket{1}) = \frac{1}{\sqrt{32}} ( \ket{00000} - \ket{00001} - \ket{00010} + \ket{00100} - ... + \ket{11111} )$
\end{eg}
